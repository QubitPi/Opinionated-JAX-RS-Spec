\documentclass[12pt]{book}

\usepackage[margin=1.5cm]{geometry}

% font
\usepackage{fontspec}
\setmainfont{Ubuntu}
\setmonofont{Ubuntu Mono}
\setsansfont{Ubuntu}

\usepackage{float}
\usepackage{graphicx}

\usepackage{parskip}  % remove indentation
\usepackage{listings} % code listings

\usepackage[colorlinks=true,urlcolor=blue,linkcolor=red]{hyperref}
\urlstyle{same} % Change \url{…} font to the main font instead of mono font

\usepackage{xcolor}
\definecolor{highlight green}{HTML}{00AA00}



%\usepackage[hyperref, thmmarks]{styles/ntheorem}
%\usepackage{framed}
%\usepackage{times}
%\usepackage{textcomp}
%\usepackage{tabularx}
%\usepackage{moreverb}
%\usepackage[pdftex]{graphicx}
%\usepackage{amssymb}
%\usepackage{float}
%\usepackage[pdftex,
%  pdftitle={JAX-RS: Java API for RESTful Web Services},
%  pdfauthor={Santiago Pericas-Geertsen and Pavel Bucek, Oracle},
%  pdfsubject={JAX-RS: Java API for RESTful Web Services},
%  pdfkeywords={Java XML Web Services API REST RESTful},
%  pdftex,
%  colorlinks=true,
%  linkcolor=black,
%  citecolor=black,
%  pdfstartview=FitH,
%  letterpaper=true,
%  bookmarksnumbered=true
%]{hyperref}
%
%% set up page dimensions
%\textwidth = 6.5 in
%\textheight = 9.33 in
%\oddsidemargin = 0.0 in
%\evensidemargin = 0.0 in
%\topmargin = -0.75 in
%\headheight = 15pt
%\headsep = 0.33 in
%\parskip = 6pt
%\parindent = 0.0in
%\footskip=0.5in
%
%\usepackage{fancyhdr}
%\pagestyle{fancy}
%\usepackage{styles/jsr}
%\theoremstyle{plain}
%%\setlength{\theorempostskipamount}{0pt}
%\newtheorem{ednote}{Editors Note}[chapter]
%
%\theoremstyle{nonumberplain}
%\theoremindent0cm
%\theoremseparator{:}
%\newtheorem{nnnote}{Note}[chapter]
%
%%\usepackage[pagewise, right]{styles/lineno}
%%\usepackage[dvips]{changebar}
%\usepackage{longtable}
%
%%\renewcommand{\today}{September 8, 2008}

\begin{document}

%\frontmatter
%\include{styles/macros}
%\include{styles/jsrfrontstyle}
%
%\begin{titlepage}
\raggedleft

\vspace*{60pt}

{\Huge
\textsf{JAX-RS: Java\texttrademark\ API for RESTful\\\vspace{10pt}
 Web Services}}

\end{titlepage} 

%
%\cleardoublepage
%{
%\small
%%\include{chapters/license}
%%\include{chapters/license_final}
%}
%
%\pdfbookmark[0]{Contents}{toc}
%\tableofcontents
%
%\mainmatter
%\include{styles/jsrmainstyle}
%
%%\linenumbers
%%\nochangebars
%
%\setcounter{secnumdepth}{4}
%
%\newcommand{\footnoteremember}[2]{\footnote{#2}\newcounter{#1}\setcounter{#1}{\value{footnote}}}
%\newcommand{\footnoterecall}[1]{\footnotemark[\value{#1}]}

    \begin{titlepage}
\raggedleft

\vspace*{60pt}

{\Huge
\textsf{JAX-RS: Java\texttrademark\ API for RESTful\\\vspace{10pt}
 Web Services}}

\end{titlepage} 


    \tableofcontents

    \chapter{Introduction}

This specification defines a set of Java APIs for the development of Web services built according to the
Representational State Transfer (REST) architectural style.

\section{Status}
\label{status}

This is the final release of version 2.1. The issue tracking system for this release can be found at
\url{https://github.com/jJAX-RS/api/issues}

The corresponding Javadocs can be found online at \url{https://jJAX-RS.github.io/apidocs/2.1}

\textcolor{highlight green}{The reference implementation can be obtained from \url{https://jersey.github.io}}
\footnote{This is one advantage of Jersey over Spring, i.e. portability}

\section{Goals}

The following are the goals of the API:

\begin{description}
    \item[POJO-based] The API will provide a set of annotations and associated classes/interfaces that may be used with
    POJOs in order to expose them as Web resources. The specification will define object lifecycle and scope.
    \item[HTTP-centric] The specification will assume HTTP is the underlying network protocol and will provide a clear
    mapping between HTTP and URI elements and the corresponding API classes and annotations. The API will provide high
    level support for common HTTP usage patterns and will be sufficiently flexible to support a variety of HTTP
    applications including WebDAV and the Atom Publishing Protocol.
    \item[Format independence] The API will be applicable to a wide variety of HTTP entity body content types. It will
    provide the necessary pluggability to allow additional types to be added by an application in a standard manner.
    \item[Container independence] Artifacts using the API will be deployable in a variety of Web-tier containers. The
    specification will define how artifacts are deployed in a Servlet container and as a JAX-WS Provider.
    \item[Inclusion in Java EE] The specification will define the environment for a Web resource class hosted in a Java
    EE container and will specify how to use Java EE features and components within a Web resource class.
\end{description}

\section{Non-Goals}
\label{non_goals}

The following are non-goals:

\begin{description}
    \item[Support for Java versions prior to Java SE 8] The API will make extensive use of annotations and lambda
    expressions that require Java SE 8 or later.
    \item[Description, registration and discovery] The specification will neither define nor require any service
    description, registration or discovery capability.
    \item[HTTP Stack] The specification will not define a new HTTP stack. HTTP protocol support is provided by a
    container that hosts artifacts developed using the API.
    \item[Data model/format classes] The API will not define classes that support manipulation of entity body content,
    rather it will provide pluggability to allow such classes to be used by artifacts developed using the API.
\end{description}

\section{Terminology}
\label{terminology}

\begin{description}
    \item[Resource class] A Java class that uses JAX-RS annotations to implement a corresponding Web resource, see
    Chapter \ref{resources}.
    \item[Root resource class] A resource class annotated with \lstinline{@Path}. Root resource classes provide the
    roots of the resource class tree and provide access to sub-resources, see Chapter \ref{resources}.
    \item[Request method designator] A runtime annotation annotated with \lstinline{@HttpMethod}. Used to identify the
    HTTP request method to be handled by a resource method.
    \item[Resource method] A method of a resource class annotated with a request method designator that is used to
    handle requests on the corresponding resource, see Section \ref{resource_method}.
    \item[Sub-resource locator] A method of a resource class that is used to locate sub-resources of the corresponding
    resource, see Section \ref{sub_resources}.
    \item[Sub-resource method] A method of a resource class that is used to handle requests on a sub-resource of the
    corresponding resource, see Section \ref{sub_resources}.
    \item[Provider] An implementation of a JAX-RS extension interface. Providers extend the capabilities of a JAX-RS
    runtime and are described in Chapter \ref{providers}.
    \item[Filter] A provider used for filtering requests and responses.
    \item[Entity Interceptor] A provider used for intercepting calls to message body readers and writers.
    \item[Invocation] A Client API object that can be configured to issue an HTTP request.
    \item[WebTarget] The recipient of an Invocation, identified by a URI.
    \item[Link] A URI with additional meta-data such as a media type, a relation, a title, etc.
\end{description}

    \chapter{Applications}
\label{applications}

A JAX-RS application consists of one or more resources (see Chapter \ref{resources}) and zero or more providers
(see Chapter \ref{providers}).

\section{Configuration}
\label{config}

\textcolor{highlight green}{The resources and providers that make up a JAX-RS application are configured via an
application-supplied subclass of \lstinline{Application}}. An implementation MAY provide alternate mechanisms for
locating resource classes and providers (e.g. runtime class scanning) but use of \lstinline{Application} is the only
portable means of configuration.

\section{Verification}
\label{verification}

Specific application requirements are detailed throughout this specification and the JAX-RS Javadocs. Implementations
MAY perform verification steps that go beyond what it is stated in this document.

A JAX-RS implementation MAY report an error condition if it detects that two or more resources could result in an
ambiguity during the execution of the algorithm described Section \ref{request_matching}. For example, if two resource
methods in the same resource class have identical (or even intersecting) values in all the annotations that are relevant
to the algorithm described in that section. The exact set of verification steps as well as the error reporting mechanism
is implementation dependent.

\section{Publication}

Applications are published in different ways depending on whether the application is run in a Java SE environment or
within a container. This section describes the alternate means of publication.

\subsection{Java SE}

In a Java SE environment a configured instance of an endpoint class can be obtained using the createEndpoint method of
RuntimeDelegate. The application supplies an instance of \lstinline{Application} and the type of endpoint required. An
implementation MAY support zero or more endpoint types of any desired type.

How the resulting endpoint class instance is used to publish the application is outside the scope of this specification.

\subsubsection{JAX-WS}

An implementation that supports publication via JAX-WS MUST support createEndpoint with an endpoint type of
javax.xml.ws.Provider. JAX-WS describes how a \lstinline{Provider} based endpoint can be published in an SE environment.

\subsection{Servlet}
\label{servlet}

A JAX-RS application is packaged as a Web application in a \textcolor{highlight green}{\lstinline{.war}} file. The
application classes are packaged in \textcolor{highlight green}{\lstinline{WEB-INF/classes}} or
\textcolor{highlight green}{\lstinline{WEB-INF/lib}} and required libraries are packaged in
\textcolor{highlight green}{\lstinline{WEB-INF/lib}}. See the Servlet specification for full details on packaging of web
applications.

    \chapter{Resources}
\label{resources}

Using JAX-RS, \textcolor{highlight green}{a Web resource is implemented as a resource class and requests are handled by
resource methods}\footnote{Endpoints should exists in a package called "resource" for readability}.

\section{Resource Classes}

\textcolor{highlight green}{A resource class is a Java class that uses JAX-RS annotations to implement a corresponding
Web resource. Resource classes are POJOs that have at least one method annotated with \lstinline{@Path} or a request method
designator}.

\subsection{Lifecycle and Environment}

\textcolor{highlight green}{By default a new resource class instance is created for each request} to that resource.
First the constructor (see Section \ref{resource_class_constructor}) is called, then any requested dependencies are
injected (see Section \ref{resource_field}), then the appropriate method (see Section \ref{resource_method}) is invoked
and finally the object is made available for garbage collection.

\subsection{Constructors}
\label{resource_class_constructor}

Root resource classes are instantiated by the JAX-RS runtime and MUST have a public constructor for which the JAX-RS
runtime can provide all parameter values. Note that a zero argument constructor is permissible under this rule.

A public constructor MAY include parameters annotated with one of the following: \lstinline{@Context},
\lstinline{@HeaderParam}, \lstinline{@CookieParam}, \lstinline{@MatrixParam}, \lstinline{@QueryParam} or
\lstinline{@PathParam}. However, depending on the resource class lifecycle and concurrency, per-request information may
not make sense in a constructor. If more than one public constructor is suitable then an implementation MUST use the one
with the most parameters. Choosing amongst suitable constructors with the same number of parameters is implementation
specific, implementations SHOULD generate a warning about such ambiguity.

Non-root resource classes are instantiated by an application and do not require the above-described public constructor.

\section{Fields and Bean Properties}
\label{resource_field}

When a resource class is instantiated, the values of fields and bean properties annotated with one the following
annotations are set according to the semantics of the annotation:

\begin{description}
    \item[\lstinline{@MatrixParam}] Extracts the value of a URI matrix parameter.
    \item[\lstinline{@QueryParam}] Extracts the value of a URI query parameter.
    \item[\lstinline{@PathParam}] Extracts the value of a URI template parameter.
    \item[\lstinline{@CookieParam}] Extracts the value of a cookie.
    \item[\lstinline{@HeaderParam}] Extracts the value of a header.
    \item[\lstinline{@Context}] Injects an instance of a supported resource, see chapters \ref{context} and
    \ref{environment} for more details.
\end{description}

Because injection occurs at object creation time, use of these annotations (with the exception of \lstinline{@Context})
on resource class fields and bean properties is only supported for the default per-request resource class lifecycle. An
implementation SHOULD warn if resource classes with other lifecycles use these annotations on resource class fields or
bean properties.

A JAX-RS implementation is only required to set the annotated field and bean property values of instances created by its
runtime. Objects returned by sub-resource locators (see Section \ref{sub_resources}) are expected to be initialized by
their creator.

Valid parameter types for each of the above annotations are listed in the corresponding Javadoc, however in general
(excluding \lstinline{@Context}) the following types are supported:

\begin{enumerate}
    \item\label{paramconverter} Types for which a ParamConverter is available via a registered ParamConverterProvider.
    See Javadoc for these classes for more information.
    \item Primitive types.
    \item\label{stringctor} Types that have a constructor that accepts a single \lstinline{String} argument.
    \item\label{valueofmethod} Types that have a static method named \lstinline{valueOf} or \lstinline{fromString} with
    a single \lstinline{String} argument that return an instance of the type. If both methods are present then
    \lstinline{valueOf} MUST be used unless the type is an enum in which case \lstinline{fromString} MUST be
    used\footnote{Due to limitations of the built-in \lstinline{valueOf} method that is part of all Java enumerations, a
    \lstinline{fromString} method is often defined by the enum writers. Consequently, the \lstinline{fromString} method
    is preferred when available.}.
    \item \lstinline{List<T>}, \lstinline{Set<T>}, or \lstinline{SortedSet<T>}, where \lstinline{T} satisfies
    \ref{paramconverter}, \ref{stringctor} or \ref{valueofmethod} above.
\end{enumerate}

The \lstinline{DefaultValue} annotation may be used to supply a default value for some of the above, see the Javadoc for
\lstinline{DefaultValue} for usage details and rules for generating a value in the absence of this annotation and the
requested data. The \lstinline{Encoded} annotation may be used to disable automatic URI decoding for
\lstinline{@MatrixParam}, \lstinline{@QueryParam}, and \lstinline{@PathParam} annotated fields and properties.

A WebApplicationException thrown during construction of field or property values using any of the 5 steps listed above
is processed directly as described in Section \ref{method_exc}. Other exceptions thrown during construction of field or
property values using any of the 5 steps listed above are treated as client errors: if the field or property is
annotated with \lstinline{@MatrixParam}, \lstinline{@QueryParam}, and \lstinline{@PathParam} then an implementation MUST
generate an instance of \lstinline{NotFoundException} (404 status) that wraps the thrown exception and no entity; if the
field or property is annotated with \lstinline{@HeaderParam} or \lstinline{@CookieParam} then an implementation MUST
generate an instance of \lstinline{BadRequestException} (400 status) that wraps the thrown exception and no entity.
Exceptions MUST be processed as described in Section \ref{method_exc}.

\section{Resource Methods}
\label{resource_method}

Resource methods are methods of a resource class annotated with a request method designator. They are used to handle
requests and MUST conform to certain restrictions described in this section.

A request method designator is a runtime annotation that is annotated with the \lstinline{@HttpMethod} annotation.
JAX-RS defines a set of request method designators for the common HTTP methods: \lstinline{@GET}, \lstinline{@POST},
\lstinline{@PUT}, \lstinline{@DELETE}, \lstinline{@PATCH}, \lstinline{@HEAD} and \lstinline{@OPTIONS}. Users may define
their own custom request method designators including alternate designators for the common HTTP methods.

\subsection{Visibility}
\label{visibility}

Only \lstinline{public} methods may be exposed as resource methods. An implementation SHOULD warn users if a non-public
method carries a method designator or \lstinline{@Path} annotation.

\subsection{Parameters}
\label{resource_method_params}

When a resource method is invoked, parameters annotated with \lstinline{@FormParam} or one of the annotations listed in
Section \ref{resource_field} are mapped from the request according to the semantics of the annotation. Similar to fields
and bean properties:

\begin{itemize}
    \item The DefaultValue annotation may be used to supply a default value for parameters
    \item The Encoded annotation may be used to disable automatic URI decoding of parameter values
    \item Exceptions thrown during construction of parameter values are treated the same as exceptions thrown during
    construction of field or bean property values, see Section \ref{resource_field}. Exceptions thrown during
    construction of \lstinline{@FormParam} annotated parameter values are treated the same as if the parameter were
    annotated with \lstinline{@HeaderParam}.
\end{itemize}

\subsubsection{Entity Parameters}
\label{entity_parameters}

The value of a parameter not annotated with \lstinline{@FormParam} or any of the annotations listed in in Section
\ref{resource_field}, called the entity parameter, is mapped from the request entity body. Conversion between an entity
body and a Java type is the responsibility of an entity provider, see Section \ref{entity_providers}. Resource methods
MUST have at most one entity parameter.

\subsection{Return Type}
\label{resource_method_return}

\textcolor{highlight green}{Resource methods MAY return \lstinline{void}, Response, GenericEntity, or another Java
type}, these return types are mapped to a response entity body as follows:

\begin{description}
    \item[\lstinline{void}] Results in an empty entity body with a 204 status code.
    \item[Response] Results in an entity body mapped from the entity property of the Response with the status code
    specified by the status property of the Response. A \lstinline{null} return value results in a 204 status code.
    If the status property of the Response is not set: a 200 status code is used for a non-null entity property and a
    204 status code is used if the entity property is \lstinline{null}.
    \item[GenericEntity] Results in an entity body mapped from the \lstinline{Entity} property of the GenericEntity. If
    the return value is not \lstinline{null} a 200 status code is used, a \lstinline{null} return value results in a 204
    status code.
    \item[Other] Results in an entity body mapped from the class of the returned instance or of its type parameter
    \lstinline{T} if the return type is \lstinline{CompletionStage<T>} (see Section \ref{completionstage}); if the class
    is an anonymous inner class, its superclass is used instead. If the return value is not \lstinline{null} a 200
    status code is used, a \lstinline{null} return value results in a 204 status code.
\end{description}

Methods that need to provide additional metadata with a response should return an instance of Response, the Response
\lstinline{Builder} class provides a convenient way to create a Response instance using a builder pattern.

\subsection{Exceptions}
\label{method_exc}

A resource method, sub-resource method or sub-resource locator may throw any checked or unchecked exception. An
implementation MUST catch all exceptions and process them in the following order:

\begin{enumerate}
    \item Instances of WebApplicationException and its subclasses MUST be mapped to a response as follows. If the
    \lstinline{response} property of the exception does not contain an entity and an exception mapping provider (see
    Section \ref{exceptionmapper}) is available for WebApplicationException or the corresponding subclass, an
    implementation MUST use the provider to create a new Response instance, otherwise the \lstinline{response} property
    is used directly. The resulting Response instance is then processed according to Section
    \ref{resource_method_return}.
    \item If an exception mapping provider (see Section \ref{exceptionmapper}) is available for the exception or one of
    its superclasses, an implementation MUST use the provider whose generic type is the nearest superclass of the
    exception to create a Response instance that is then processed according to Section \ref{resource_method_return}.
    If the exception mapping provider throws an exception while creating a Response then return a server error (status
    code 500) response to the client.
    \item\label{runtimeexc} Unchecked exceptions and errors that have not been mapped MUST be re-thrown and allowed to
    propagate to the underlying container.
    \item\label{checkedexc} Checked exceptions and throwables that have not been mapped and cannot be thrown directly
    MUST be wrapped in a container-specific exception that is then thrown and allowed to propagate to the underlying
    container. Servlet-based implementations MUST use \lstinline{Servlet\-Exception} as the wrapper. JAX-WS
    \lstinline{Provider}-based implementations MUST use \lstinline{Web\-Service\-Exception} as the wrapper.
\end{enumerate}

Note: Items \ref{runtimeexc} and \ref{checkedexc} allow existing container facilities (e.g. a Servlet filter or error
pages) to be used to handle the error if desired.

\subsection{HEAD and OPTIONS}
\label{head_and_options}

\lstinline{HEAD} and \lstinline{OPTIONS} requests receive additional automated support. On receipt of a \lstinline{HEAD}
request an implementation MUST either:

\begin{enumerate}
    \item Call a method annotated with a request method designator for \lstinline{HEAD} or, if none present,
    \item\label{get_not_head} Call a method annotated with a request method designator for \lstinline{GET} and discard
    any returned entity.
\end{enumerate}

\textcolor{highlight green}{Note that option \ref{get_not_head} may result in reduced performance where entity creation is significant.}

On receipt of an \lstinline{OPTIONS} request an implementation MUST either:

\begin{enumerate}
    \item Call a method annotated with a request method designator for \lstinline{OPTIONS} or, if none present,
    \item Generate an automatic response using the metadata provided by the JAX-RS annotations on the matching class and
    its methods.
\end{enumerate}

\section{URI Templates}
\label{uritemplates}

\textcolor{highlight green}{A root resource class is anchored in URI space using the \lstinline{@Path} annotation. The
value of the annotation is a relative URI path template whose base URI is provided by the combination of the deployment
context and the application path (see the \lstinline{@ApplicationPath} annotation)}.

A URI path template is a string with zero or more embedded parameters that, when values are substituted for all the
parameters, is a valid URI path. The Javadoc for the \lstinline{@Path} annotation describes their syntax. E.g.:

\begin{lstlisting}[language=Java]
    @Path("widgets/{id}")
    public class Widget {
        ...
    }
\end{lstlisting}

In the above example the \lstinline{Widget} resource class is identified by the relative URI path
\lstinline{widgets/xxx} where \lstinline{xxx} is the value of the \lstinline{id} parameter.

Note: Because \lq\{\rq and \lq\}\rq\ are not part of either the reserved or unreserved productions of URI they will not
appear in a valid URI.

The value of the annotation is automatically encoded, e.g. the following two lines are equivalent:

\begin{lstlisting}[language=Java]
    @Path("widget list/{id}")
    @Path("widget%20list/{id}")
\end{lstlisting}

Template parameters can optionally specify the regular expression used to match their values. The default value matches
any text and terminates at the end of a path segment but other values can be used to alter this behavior, e.g.:

\begin{lstlisting}[language=Java]
    @Path("widgets/{path:.+}")
    public class Widget {
        ...
    }
\end{lstlisting}

In the above example the \lstinline{Widget} resource class will be matched for any request whose path starts with
\lstinline{widgets} and contains at least one more path segment; the value of the \lstinline{path} parameter will be the
request path following \lstinline{widgets}. E.g. given the request path \lstinline{widgets/small/a} the value of
\lstinline{path} would be \lstinline{small/a}.

The value of a URI path parameter is available for injection via \lstinline{@PathParam} on a field, property or method
parameter. Note that if a URI template is used on a method, a path parameter injected in a field or property may not be
available (set to \lstinline{null}). The following example illustrates this scenario:

\begin{lstlisting}[language=Java]
    @Path("widgets")
    public class WidgetsResource {

        @PathParam("id") String id;

        @GET
        public WidgetList getWidgets() {
        ... // id is null here
        }

        @GET
        @Path("{id}")
        public Widget findWidget() {
        return new WidgetResource(id);
        }
    }
\end{lstlisting}

\subsection{Sub Resources}
\label{sub_resources}

\textcolor{highlight green}{Methods of a resource class that are annotated with \lstinline{@Path} are either
sub-resource methods or sub-resource locators}. Sub-resource methods handle a HTTP request directly whilst sub-resource
locators return an object or class that will handle a HTTP request. The presence or absence of a request method
designator (e.g. \lstinline{@GET}) differentiates between the two:

\begin{description}
    \item[Present] Such methods, known as {\em sub-resource methods}, are treated like a normal resource method (see
    Section \ref{resource_method}) except the method is only invoked for request URIs that match a URI template created
    by concatenating the URI template of the resource class with the URI template of the method\footnote{If the resource
    class URI template does not end with a \lq/\rq\ character then one is added during the concatenation.}.

    \item[Absent] Such methods, known as {\em sub-resource locators}, are used to dynamically resolve the object that
    will handle the request. Sub-resource locators can return objects or classes; if a class is returned then an object
    is obtained by the implementation using a suitable constructor as described in Section
    \ref{resource_class_constructor}. In either case, the resulting object is used to handle the request or to further
    resolve the object that will handle the request, see \ref{mapping_requests_to_java_methods} for further details.

    When an object is returned, implementations MUST dynamically determine its class rather than relying on the static
    sub-resource locator return type, since the returned instance may be a subclass of the declared type with
    potentially different annotations, see Section \ref{annotationinheritance} for rules on annotation inheritance.
    Sub-resource locators may have all the same parameters as a normal resource method (see Section
    \ref{resource_method}) except that they MUST NOT have an entity parameter.
\end{description}

The following example illustrates the difference:

\begin{lstlisting}[language=Java]
    @Path("widgets")
    public class WidgetsResource {

        @GET
        @Path("offers")
        public WidgetList getDiscounted() {...}

        @Path("{id}")
        public WidgetResource findWidget(@PathParam("id") String id) {
            return new WidgetResource(id);
        }
    }

    public class WidgetResource {

        public WidgetResource(String id) {...}

        @GET
        public Widget getDetails() {...}
    }
\end{lstlisting}

In the above a \lstinline{GET} request for the \lstinline{widgets/offers} resource is handled directly by the
\lstinline{getDiscounted} sub-resource method of the resource class
\lstinline{WidgetsResource} whereas a \lstinline{GET} request for \lstinline{widgets/xxx} is handled by the
\lstinline{getDetails} method of the \lstinline{WidgetResource} resource class.

Note: A set of sub-resource methods annotated with the same URI template value are functionally equivalent to a
similarly annotated sub-resource locator that returns an instance of a resource class with the same set of resource
methods.

\section{Declaring Media Type Capabilities}
\label{declaring_method_capabilities}

\textcolor{highlight green}{Application classes can declare the supported request and response media types using the
\lstinline{@Consumes} and \lstinline{@Produces} annotations respectively}. These annotations MAY be applied to a
resource method, a resource class, or to an entity provider (see Section \ref{declaring_provider_capabilities}). Use of
these annotations on a resource method overrides any on the resource class or on an entity provider for a method
argument or return type. In the absence of either of these annotations, support for any media type (\lq\lq*/*\rq\rq) is
assumed.

The following example illustrates the use of these annotations:

\begin{lstlisting}[language=Java]

    @Path("widgets")
    @Produces("application/widgets+xml")
    public class WidgetsResource {

        @GET
        public Widgets getAsXML() {...}

        @GET
        @Produces("text/html")
        public String getAsHtml() {...}

        @POST
        @Consumes("application/widgets+xml")
        public void addWidget(Widget widget) {...}
    }

    @Provider
    @Produces("application/widgets+xml")
    public class WidgetsProvider implements MessageBodyWriter<Widgets> {...}

    @Provider
    @Consumes("application/widgets+xml")
    public class WidgetProvider implements MessageBodyReader<Widget> {...}
\end{lstlisting}

In the above:
\begin{itemize}
    \item The \lstinline{getAsXML} resource method will be called for \lstinline{GET} requests that specify a response
    media type of \lstinline{application/widgets+xml}. It returns a \lstinline{Widgets} instance that will be mapped to
    that format using the \lstinline{WidgetsProvider} class (see Section \ref{entity_providers} for more information on
    MessageBodyWriter).
    \item The \lstinline{getAsHtml} resource method will be called for \lstinline{GET} requests that specify a response
    media type of \lstinline{text/html}. It returns a \lstinline{String} containing \lstinline{text/html} that will be
    written using the default implementation of \lstinline{MessageBodyWriter<String>}.
    \item The \lstinline{addWidget} resource method will be called for \lstinline{POST} requests that contain an entity
    of the media type \lstinline{application/widgets+xml}. The value of the \lstinline{widget} parameter will be mapped
    from the request entity using the \lstinline{WidgetProvider} class (see Section \ref{entity_providers} for more
    information on MessageBodyReader).
\end{itemize}

An implementation MUST NOT invoke a method whose effective value of \lstinline{@Produces} does not match the request
\lstinline{Accept} header. An implementation MUST NOT invoke a method whose effective value of \lstinline{@Consumes}
does not match the request \lstinline{Content-Type} header.

When accepting multiple media types, clients may indicate preferences by using a relative quality factor known as the q
parameter. The value of the q parameter, or q-value, is used to sort the set of accepted types. For example, a client
may indicate preference for \lstinline{application/widgets+xml} with a relative quality factor of 1 and for
\lstinline{application/xml} with a relative quality factor of 0.8. Q-values range from 0 (undesirable) to 1
(highly desirable), with 1 used as default when omitted. A \lstinline{GET} request matched to the
\lstinline{WidgetsResource} class with an accept header of \lstinline{text/html; q=1, application/widgets+xml; q=0.8}
will result in a call to method \lstinline{getAsHtml} instead of \lstinline{getAsXML} based on the value of q.

A server can also indicate media type preference using the qs parameter; server preference is only examined when
multiple media types are accepted by a client {\em with the same q-value}. Consider the following example:

\begin{lstlisting}[language=Java]
    @Path("widgets2")
    public class WidgetsResource2 {

        @GET
        @Produces("application/xml", "application/json")
        public Widgets getWidget() {...}
    }
\end{lstlisting}

Suppose a client issues a \lstinline{GET} request with an accept header of \lstinline{application/*; q=0.5, text/html}.
Based on this request, the server determines that both \lstinline{application/xml} and \lstinline{application/json} are
equally preferred by the client with a q-value of 0.5. By specifying a server relative quality factor as part of the
\lstinline{@Produces} annotation, it is possible to control which response media type to select:

\begin{lstlisting}[language=Java]
    @Path("widgets2")
    public class WidgetsResource2 {

        @GET
        @Produces("application/xml; qs=1", "application/json; qs=0.75")
        public Widgets getWidget() {...}
    }
\end{lstlisting}

With the updated value for \lstinline{@Produces} in this example, and in response to a \lstinline{GET} request with an
accept header that includes \lstinline{application/*; q=0.5}, JAX-RS implementations are REQUIRED to select the media
type \lstinline{application/xml} given its higher qs-value. Note that qs-values, just like q-values, are relative and as
such are only comparable to other qs-values within the same \lstinline{@Produces} annotation instance. For more
information see Section~\ref{determine_response_type}.

\section{Annotation Inheritance}
\label{annotationinheritance}

\textcolor{highlight green}{JAX-RS annotations may be used on the methods and method parameters of a super-class or an
implemented interface. Such annotations are inherited by a corresponding sub-class or implementation class method
provided that the method and its parameters do not have any JAX-RS annotations of their own. Annotations on a
super-class take precedence over those on an implemented interface. The precedence over conflicting annotations defined
in multiple implemented interfaces is implementation specific. Note that inheritance of class or interface annotations
is not supported.}

If a subclass or implementation method has any JAX-RS annotations then all of the annotations on the superclass or
interface method are ignored. E.g.:

\begin{lstlisting}[language=Java]
    public interface ReadOnlyAtomFeed {

        @GET
        @Produces("application/atom+xml")
        Feed getFeed();
    }

    @Path("feed")
    public class ActivityLog implements ReadOnlyAtomFeed {

        public Feed getFeed() {...}
    }
\end{lstlisting}

In the above, \lstinline{ActivityLog.getFeed} inherits the \lstinline{@GET} and \lstinline{@Produces} annotations from
the interface. Conversely:

\begin{lstlisting}[language=Java]
    @Path("feed")
    public class ActivityLog implements ReadOnlyAtomFeed {

        @Produces("application/atom+xml")
        public Feed getFeed() {...}
    }
\end{lstlisting}

In the above, the \lstinline{@GET} annotation on \lstinline{ReadOnlyAtomFeed.getFeed} is not inherited by
\lstinline{ActivityLog.getFeed} and it would require its own request method designator since it redefines the
\lstinline{@Produces} annotation.

For consistency with other Java EE specifications, it is recommended to always repeat annotations instead of relying on
annotation inheritance.

\section{Matching Requests to Resource Methods}
\label{mapping_requests_to_java_methods}

This section describes how a request is matched to a resource class and method. Implementations are not required to use the algorithm as written but MUST produce results equivalent to those produced by the algorithm.

\subsection{Request Preprocessing}
\label{reqpreproc}

Prior to matching, request URIs are normalized\footnote{Note: some containers might perform this functionality prior to
passing the request to an implementation.} by following the rules for case, path segment, and percent encoding
normalization described in section 6.2.2 of RFC 3986. The normalized request URI MUST be reflected in the URIs obtained
from an injected \lstinline{UriInfo}.

\subsection{Request Matching}
\label{request_matching}

A request is matched to the corresponding resource method or sub-resource method by comparing the
\textcolor{highlight green}{normalized} request URI (see Section \ref{reqpreproc}), the media type of any request
entity, and the requested response entity format to the metadata annotations on the resource classes and their methods.
If no matching resource method or sub-resource method can be found then an appropriate error response is returned. All
exceptions reported by this algorithm MUST be processed as described in Section \ref{method_exc}.

Matching of requests to resource methods proceeds in three stages as follows:

\begin{enumerate}
    \item Identify a set of candidate root resource classes matching the request:
    \item \label{find_object} Obtain a set of candidate resource methods for the request
    \item \label{find_method} Identify the method that will handle the request:
\end{enumerate}

    \include{chapters/providers}
    \chapter{Client API}
\label{client_api}

The Client API is used to access Web resources. It provides a higher-level API than \lstinline{HttpURLConnection} as
well as integration with JAX-RS providers. Unless otherwise stated, types presented in this chapter live in the
\lstinline{javax.ws.rs.client} package.

\section{Bootstrapping a Client Instance}

An instance of Client is required to access a Web resource using the Client API. The default instance of Client can be
obtained by calling \lstinline{newClient} on ClientBuilder. Client instances can be configured using methods inherited
from \lstinline{Configurable} as follows:

\begin{lstlisting}[language=Java]
    // Default instance of client
    Client client = ClientBuilder.newClient();

    // Additional configuration of default client
    client.property("MyProperty", "MyValue")
    .register(MyProvider.class)
    .register(MyFeature.class);
\end{lstlisting}

See Chapter \ref{providers} for more information on providers. Properties are simply name-value pairs where the value is
an arbitrary object. Features are also providers and must implement the \lstinline{Feature} interface; they are useful
for grouping sets of properties and providers (including other features) that are logically related and must be enabled
as a unit.

\section{Resource Access}
\label{resource_access}

A Web resource can be accessed using a fluent API in which method invocations are chained to build and ultimately submit
an HTTP request. The following example gets a \lstinline{text/plain} representation of the resource identified by
\lstinline{http://example.org/hello}:

\begin{lstlisting}[language=Java]
    Client client = ClientBuilder.newClient();
    Response res = client.target("http://example.org/hello")
    .request("text/plain").get();
\end{lstlisting}

Conceptually, the steps required to submit a request are the following: (i) obtain an instance of Client (ii) create a
WebTarget (iii) create a request from the WebTarget and (iv) submit a request or get a prepared Invocation for later
submission. See Section \ref{invocations} for more information on using Invocation.

Method chaining is not limited to the example shown above. A request can be further specified by setting headers,
cookies, query parameters, etc. For example:

\begin{lstlisting}[language=Java]
    Response res = client.target("http://example.org/hello")
    .queryParam("MyParam","...")
    .request("text/plain")
    .header("MyHeader", "...")
    .get();
\end{lstlisting}

See the Javadoc for the classes in the \lstinline{javax.ws.rs.client} package for more information.

\section{Client Targets}

The benefits of using a WebTarget become apparent when building complex URIs, for example by extending base URIs with
additional path segments or templates. The following example highlights these cases:

\begin{lstlisting}[language=Java]
    WebTarget base = client.target("http://example.org/");
    WebTarget hello = base.path("hello").path("{whom}");
    Response res = hello.resolveTemplate("whom", "world").request("...").get();
\end{lstlisting}

Note the use of the URI template parameter~\lstinline{whom}. The example above gets a representation for the resource
identified by \lstinline{http://example.org/hello/world}.

\textcolor{highlight green}{WebTarget instances are immutable with respect to their URI (or URI template)}: methods for
specifying additional path segments and parameters return a new instance of WebTarget. However,
\textcolor{highlight green}{WebTarget instances are mutable with respect to their configuration}. Thus, configuring a
WebTarget does not create new instances.

\begin{lstlisting}[language=Java]
    // Create WebTarget instance base
    WebTarget base = client.target("http://example.org/");

    // Create new WebTarget instance hello and configure it
    WebTarget hello = base.path("hello");
    hello.register(MyProvider.class);
\end{lstlisting}

In this example, two instances of WebTarget are created. The instance \lstinline{hello} inherits the configuration from
\lstinline{base} and it is further configured by registering \lstinline{MyProvider.class}. Note that changes to
\lstinline{hello}'s configuration do not affect \lstinline{base}, i.e.~inheritance performs a deep copy of the
configuration. See Section \ref{configurable_types} for additional information on configurable types.

\section{Typed Entities}

The response to a request is not limited to be of type Response. The following example upgrades the status of customer
number 123 to ``gold status'' by first obtaining an entity of type \lstinline{Customer} and then posting that entity to
a different URI:

\begin{lstlisting}[language=Java]
    Customer c = client.target("http://examples.org/customers/123")
    .request("application/xml").get(Customer.class);
    String newId = client.target("http://examples.org/gold-customers/")
    .request().post(xml(c), String.class);
\end{lstlisting}

Note the use of the variant \lstinline{xml()} in the call to \lstinline{post}. The class
\lstinline{javax.ws.rs.client.Entity} defines variants for the most popular media types used in JAX-RS applications.

In the example above, just like in the Server API, JAX-RS implementations are REQUIRED to use entity providers to map a
representation of type \lstinline{application/xml} to an instance of \lstinline{Customer} and vice versa. See Section
\ref{standard_entity_providers} for a list of entity providers that MUST be supported by all JAX-RS implementations.

\section{Invocations}
\label{invocations}

An invocation is a request that has been prepared and is ready for execution. Invocations provide a generic interface
that enables a separation of concerns between the creator and the submitter. In particular, the submitter does not need
to know how the invocation was prepared, but only how it should be executed: namely, synchronously or asynchronously.

Let us consider the following example\footnote{The Collections class in this example is arbitrary and does not
correspond to any specific implementation. There are a number of Java collection libraries available that provide this
type of functionality.}:

\begin{lstlisting}[language=Java]
    // Executed by the creator
    Invocation inv1 = client.target("http://examples.org/atm/balance")
    .queryParam("card", "111122223333")
    .queryParam("pin", "9876")
    .request("text/plain").buildGet();

    Invocation inv2 = client.target("http://examples.org/atm/withdrawal")
    .queryParam("card", "111122223333")
    .queryParam("pin", "9876")
    .request()
    .buildPost(text("50.0"));

    Collection<Invocation> invs = Arrays.asList(inv1, inv2);

    // Executed by the submitter
    Collection<Response> ress = Collections.transform(
    invs,
    new F<Invocation, Response>() {
    @Override
    public Response apply(Invocation inv) {
    return inv.invoke();
    }
    }
    );
\end{lstlisting}

In this example, two invocations are prepared and stored in a collection by the creator. The submitter then traverses
the collection applying a transformation that maps an Invocation to a Response. The mapping calls
\lstinline{Invocation.invoke} to execute the invocation synchronously; asynchronous execution is also supported by
calling \lstinline{Invocation.submit}. Refer to Chapter~\ref{asynchronous_processing} for more information on
asynchronous invocations.

\section{Configurable Types}
\label{configurable_types}

The following Client API types are configurable: Client, ClientBuilder, and WebTarget. Configuration methods are
inherited from the \lstinline{Configurable} interface implemented by all these classes. This interface supports
configuration of:

\begin{description}
    \item [Properties] Name-value pairs for additional configuration of features or other components of a JAX-RS
    implementation.
    \item [Features] A special type of provider that implement the \lstinline{Feature} interface and can be used to
    configure a JAX-RS implementation.
    \item [Providers] Classes or instances of classes that implement one or more of the provider interfaces from
    Chapter \ref{providers}. A provider can be a message body reader, a filter, a context resolver, etc.
\end{description}

The configuration defined on an instance of any of the aforementioned types is inherited by other instances created from
it. For example, an instance of WebTarget created from a Client will inherit the Client's configuration. However, any
additional changes to the instance of WebTarget will not impact the Client's configuration and vice versa. Therefore,
once a configuration is inherited it is also detached (deep copied) from its parent configuration and changes to the
parent and child configurations are not be visible to each other.

\subsection{Filters and Entity Interceptors}
\label{filters_interceptors_client}

As explained in Chapter \ref{filters_and_interceptors}, filters and interceptors are defined as JAX-RS providers.
Therefore, they can be registered in any of the configurable types listed in the previous section. The following example
shows how to register filters and interceptors on instances of Client and WebTarget:

\begin{lstlisting}[language=Java]
    // Create client and register logging filter
    Client client = ClientBuilder.newClient().register(LoggingFilter.class);

    // Executes logging filter from client and caching filter from target
    WebTarget wt = client.target("http://examples.org/customers/123");
    Customer c = wt.register(CachingFilter.class).request("application/xml").get(Customer.class);
\end{lstlisting}

In this example, \lstinline{LoggingFilter} is inherited by each instance of WebTarget created from \lstinline{client};
an additional provider named \lstinline{CachingFilter} is registered on the instance of WebTarget.

\section{Reactive Clients}
\label{reactive_clients}

Section \ref{client_api_async} introduces asynchronous programming in the Client API. Asynchronous programming in JAX-RS
enables clients to unblock certain threads by pushing work to background threads which can be monitored and possibly
waited on (joined) at a later time. This can be accomplished in JAX-RS by either providing an instance of
\lstinline{InvocationCallback} or operating on the result of type \lstinline{Future<T>} returned by an asynchronous
invoker -- or some combination of both styles.

Using \lstinline{InvocationCallback} enables a more reactive programming style in which user-provided code activates (or
reacts) only when a certain event has occurred. Using callbacks works well for simple cases, but the source code becomes
harder to understand when multiple events are in play. For example, when asynchronous invocations need to be composed,
combined or in any way operated upon. These type of scenarios may result in callbacks that are nested inside other
callbacks making the code far less readable -- often referred to as the "pyramid of doom" because of the inherent
nesting of calls.

To address the requirement of greater readability and to enable programmers to better reason about asynchronous
computations, Java 8 introduces the a new interface called \lstinline{CompletionStage} that includes a large number of
methods dedicated to managing asynchronous computations.

JAX-RS 2.1 defines a new type of invoker called \lstinline{RxInvoker}, as well a default implementation of this type
called \lstinline{CompletionStageRxInvoker} that is based on the Java 8 type \lstinline{CompletionStage}. There is a
new \lstinline{rx} method which is used in a similar manner to \lstinline{async} as described in \ref{client_api_async}.
Let us consider the following example:

\begin{lstlisting}[language=Java]
    CompletionStage<String> csf = client.target("forecast/{destination}")
            .resolveTemplate("destination", "mars")
            .request()
            .rx()
            .get(String.class);

    csf.thenAccept(System.out::println);
\end{lstlisting}

This example first creates an asynchronous computation of type \lstinline{CompletionStage<String>}, and then simply
waits for it to complete and displays its result (technically, a second computation of type
\lstinline{CompletionStage<Void>} is created on the last line simply to consume the result of the first computation).

The value of \lstinline{CompletionStage} becomes apparent when multiple asynchronous computations are necessary to
accomplish a task. The following example obtains, in parallel, a price and a forecast for a destination and makes a
reservation only if the desired conditions are met.

\begin{lstlisting}[language=Java]
    CompletionStage<Number> csp = client.target("price/{destination}")
            .resolveTemplate("destination", "mars")
            .request()
            .rx()
            .get(Number.class);

    CompletionStage<String> csf = client.target("forecast/{destination}")
            .resolveTemplate("destination", "mars")
            .request()
            .rx()
            .get(String.class);

    csp.thenCombine(csf, (price, forecast) -> reserveIfAffordableAndWarm(price, forecast));
\end{lstlisting}

Note that the \lstinline{Consumer} passed in the call to method \lstinline{thenCombine} requires the values of each
stage to be available and, therefore, can only be executed after the two parallel stages are completed.

As we shall see in the next section, support for \lstinline{CompletionStage} is the default for all JAX-RS
implementations, but other reactive APIs may also be supported as extensions.

\subsection{Reactive API Extensions}
\label{reactive_api_extensions}

There have been several proposals for reactive APIs in Java. All JAX-RS implementations MUST support an invoker for
\lstinline{CompletionStage} as shown above. Additionally, JAX-RS implementations MAY support other reactive APIs using
an extension built into the Client API.

RxJava is a popular reactive library available in Java. The type representing an asynchronous computation in this API
is called an \lstinline{Observable}. An implementation may support this type by providing a new invoker as shown in the following example:

\begin{lstlisting}[language=Java]
    Client client = client.register(ObservableRxInvokerProvider.class);

    Observable<String> of = client.target("forecast/{destination}")
            .resolveTemplate("destination", "mars")
            .request()
            .rx(ObservableRxInvoker.class)    // overrides default invoker
            .get(String.class);

    of.subscribe(System.out::println);
\end{lstlisting}

First, a provider for the new invoker must be registered on the \lstinline{Client} object. Second, the type of the
invoker must be specified as a parameter to the \lstinline{rx} method. Note that because this is a JAX-RS extension,
the actual names for the provider and the invoker in the example above are implementation dependent. The reader is
referred to the documentation for the JAX-RS implementation of choice for more information.

Version 2.0 of RxJava has been completely re-written on top of the Reactive-Streams specification. This new architecture
prompted the introduction of a new type called \lstinline{Flowable}. JAX-RS implementations could easily support this
new version by implementing a new provider (such as \lstinline{FlowableRxInvokerProvider}) and using the same pattern
shown in the example above.

\section{Executor Services}
\label{executor_services}

Executor services can be used to submit asynchronous tasks for execution. JAX-RS applications can specify executor
services while building a \lstinline{Client} instance. Two methods are provided in \lstinline{ClientBuilder} for this
purpose, namely, \lstinline{executorService} and \lstinline{scheduledExecutorService}.

In an environment that supports the Concurrency Utilities for Java EE, such as the Java EE Full Profile, implementations
MUST use \lstinline{ManagedExecutorService} and \lstinline{ManagedScheduledExecutorService}, respectively. The reader is
referred to the Javadoc of \lstinline{ClientBuilder} for more information about executor services.

    \include{chapters/filters}
    \chapter{Validation}
\label{validation}

Validation is the process of verifying that some data obeys one or more pre-defined constraints. The Bean Validation
specification defines an API to validate Java Beans. This chapter describes how JAX-RS provides native support for
validating resource classes. See Section \ref{bv_support} for more information on implementation requirements.

\section{Constraint Annotations}
\label{constraint_annotations}

The Server API provides support for extracting request values and mapping them into Java fields, properties and
parameters using annotations such as @HeaderParam, @QueryParam, etc. It also supports mapping of
request entity bodies into Java objects via non-annotated parameters (i.e., parameters without any JAX-RS annotations).
See Chapter \ref{resources} for additional information.

In earlier versions of JAX-RS, any additional validation of these values needed to be performed programmatically. This
version of JAX-RS introduces support for declarative validation based on the Bean Validation specification.

The Bean Validation specification supports the use of constraint annotations as a way of declaratively validating beans,
method parameters and method returned values. For example, consider the following resource class augmented with
constraint annotations:

\begin{lstlisting}[language=Java]
    @Path("/")
    class MyResourceClass {

    @POST
    @Consumes("application/x-www-form-urlencoded")
    public void registerUser(
    @NotNull @FormParam("firstName") String firstName,
    @NotNull @FormParam("lastName") String lastName,
    @Email @FormParam("email") String email) {
    ...
    }
    }
\end{lstlisting}

The annotations @NotNull and @Email impose additional constraints on the form parameters
firstName, lastName and email. The @NotNull constraint is built-in to
the Bean Validation API; the @Email constraint is assumed to be user defined in the example above.
These constraint annotations are not restricted to method parameters, they can be used in any location in which the
JAX-RS binding annotations are allowed with the exception of constructors and property setters. Rather than using method
parameters, the MyResourceClass shown above could have been written as follows:

\begin{lstlisting}[language=Java]
    @Path("/")
    class MyResourceClass {

        @NotNull
        @FormParam("firstName")
        private String firstName;

        @NotNull
        @FormParam("lastName")
        private String lastName;

        private String email;

        @FormParam("email")
        public void setEmail(String email) {
            this.email = email;
        }

        @Email
        public String getEmail() {
            return email;
        }

        ...
    }
\end{lstlisting}

Note that in this version, firstName and lastName are fields initialized via injection and
email is a resource class property. Constraint annotations on properties are specified in their
corresponding getters.

Constraint annotations are also allowed on resource classes. In addition to annotating fields and properties, an
annotation can be defined for the entire class. Let us assume that @NonEmptyNames validates that one of the
two name fields in MyResourceClass is provided. Using such an annotation, the example above can be extended
as follows:

\begin{lstlisting}[language=Java]
    @Path("/")
    @NonEmptyNames
    class MyResourceClass {

        @NotNull
        @FormParam("firstName")
        private String firstName;

        @NotNull
        @FormParam("lastName")
        private String lastName;

        private String email;

        ...
    }
\end{lstlisting}

Constraint annotations on resource classes are useful for defining cross-field and cross-property constraints.

\section{Annotations and Validators}
\label{annotations_and_validators}

Annotation constraints and validators are defined in accordance with the Bean Validation specification. The
@Email annotation shown above is defined using the Bean Validation @Constraint
meta-annotation:

\begin{lstlisting}[language=Java]
    @Retention(RUNTIME)
    @Target( { METHOD, FIELD, PARAMETER })
    @Constraint(validatedBy = EmailValidator.class)
    public @interface Email {

        String message() default "{com.example.validation.constraints.email}";
        Class<?>[] groups() default {};
        Class<? extends Payload>[] payload() default {};
    }
\end{lstlisting}

The @Constraint annotation must include a reference to the validator class that will be used to validate
decorated values. The \lstinline{EmailValidator} class must implement \lstinline{ConstraintValidator<Email, T>} where
\lstinline{T} is the type of values being validated. For example,

\begin{lstlisting}[language=Java]
    public class EmailValidator implements ConstraintValidator<Email, String> {

        public void initialize(Email email) {
            ...
        }

        public boolean isValid(String value, ConstraintValidatorContext context) {
            ...
        }
    }
\end{lstlisting}

Thus, \lstinline{EmailValidator} applies to values annotated with @Email that are of type
\lstinline{String}. Validators for different types can be defined for the same constraint annotation.

Constraint annotations must also define a \lstinline{groups} element to indicate which processing groups they are
associated with. If no groups are specified (as in the example above) the \lstinline{Default} group is assumed. For
simplicity, JAX-RS implementations are NOT REQUIRED to support processing groups other than \lstinline{Default}. In what
follows, we assume that constraint validation is carried out in the \lstinline{Default} processing group.

\section{Entity Validation}
\label{entity_validation}

Request entity bodies can be mapped to resource method parameters. There are two ways in which these entities can be
validated. If the request entity is mapped to a Java bean whose class is decorated with Bean Validation annotations,
then validation can be enabled using \lstinline{@Valid}:

\begin{lstlisting}[language=Java]
    @StandardUser
    class User {

        @NotNull
        private String firstName;
        ...
        }

    @Path("/")
    class MyResourceClass {

        @POST
        @Consumes("application/xml")
        public void registerUser(@Valid User user) {
            ...
        }
    }
\end{lstlisting}

In this case, the validator associated with \lstinline{@StandardUser} (as well as those for non-class level constraints
like \lstinline{@NotNull}) will be called to verify the request entity mapped to \lstinline{user}. Alternatively, a new
annotation can be defined and used directly on the resource method parameter.

\begin{lstlisting}[language=Java]
    @Path("/")
    class MyResourceClass {

        @POST
        @Consumes("application/xml")
        public void registerUser(@PremiumUser User user) {
            ...
        }
    }
\end{lstlisting}

In the example above, \lstinline{@PremiumUser} rather than \lstinline{@StandardUser} will be used to validate the
request entity. These two ways in which validation of entities can be triggered can also be combined by including
\lstinline{@Valid} in the list of constraints. The presence of \lstinline{@Valid} will trigger validation of all the
constraint annotations decorating a Java bean class. This validation will take place in the \lstinline{Default}
processing group unless the \lstinline{@ConvertGroup} annotation is present. See for more information on
\lstinline{@ConvertGroup}.

Response entity bodies returned from resource methods can be validated in a similar manner by annotating the resource
method itself. To exemplify, assuming both \lstinline{@StandardUser} and \lstinline{@PremiumUser} are required to be
checked before returning a user, the \lstinline{getUser} method can be annotated as shown next:

\begin{lstlisting}[language=Java]
    @Path("/")
    class MyResourceClass {

        @GET
        @Valid
        @PremiumUser
        @Path("{id}")
        @Produces("application/xml")
        public User getUser(@PathParam("id") String id) {
            User u = findUser(id);
            return u;
        }
        ...
    }
\end{lstlisting}

Note that \lstinline{@PremiumUser} is explicitly listed and \lstinline{@StandardUser} is triggered by the presence of
the \lstinline{@Valid}\ annotation -- see definition of \lstinline{User} class earlier in this section.

\section{Default Validation Mode}
\label{default_validation_mode}

Validation is enabled by default only for the so called constrained methods. Getter methods as defined by the Java
Beans specification are not constrained methods, so they will not be validated by default. The special annotation
\lstinline{@ValidateOnExecution} can be used to selectively enable and disable validation. For example, you can enable
validation on method \lstinline{getEmail} shown above as follows:

\begin{lstlisting}[language=Java]
    @Path("/")
    class MyResourceClass {

        @Email
        @ValidateOnExecution
        public String getEmail() {
            return email;
        }

        ...
    }
\end{lstlisting}

The default value for the \lstinline{type} attribute of \lstinline{@ValidateOnExecution} is \lstinline{IMPLICIT} which,
in the example above, results in method \lstinline{getEmail} being validated. See \cite{bv11} for more information on
other uses of this annotation.

Note that if validation for getter methods is enabled and a resource method's signature obeys the rules for getters, the
resource method may be (unintentionally) invoked during validation. Conversely, if validation for getter methods is
disabled and the matching resource method's signature obeys the rules for getters, the JAX-RS runtime will still
validate the method (i.e., the validation preference will be ignored) before invocation.

\section{Annotation Inheritance}
\label{annotation_inheritance}

    It is worth noting that these rules are incompatible with those defined in Section \ref{annotationinheritance}.
    Generally speaking, constraint annotations in are cumulative (can be strengthen) across a given type hierarchy while
    JAX-RS annotations are inherited or, overridden and ignored.

    The goal of this specification is to enable validation of JAX-RS resources by leveraging existing Bean Validation
    implementations.

    \section{Validation and Error Reporting}
    \label{validation_and_error_reporting}

    Constraint annotations are allowed in the same locations as the following annotations: \lstinline{@MatrixParam},
    \lstinline{@QueryParam}, \lstinline{@PathParam}, \lstinline{@CookieParam}, \lstinline{@HeaderParam} and
    \lstinline{@Context}, except in class constructors and property setters. Specifically, they are allowed in resource
    method parameters, fields and property getters as well as resource classes, entity parameters and resource methods
    (return values).

    The default resource class instance lifecycle is per-request in JAX-RS. Implementations MAY support other lifecycles;
    the same caveats related to the use of other JAX-RS annotations in resource classes apply to constraint annotations. For
    example, a constraint validation annotating a constructor parameter in a resource class whose lifecycle is singleton
    will only be executed once.

    JAX-RS implementations SHOULD use the following process to validate resource class instances after they have been
    instantiated:

\begin{description}
    \item[Phase 1] Inject field values and initialize bean properties as described in Section \ref{resource_field}.
    \item[Phase 2] Validate annotations on fields, property getters (if enabled) and the resource class. The order in
    which these validations are executed is implementation dependent.
    \item[Phase 3] Validate annotations on parameters passed to the resource method matched.
    \item[Phase 4] If no constraint violations found thus far, invoke resource method and validate returned value.
\end{description}

The exception model defines a base class javax.validation.ValidationException and a few subclasses to report errors that
are specific to constraint definitions, constraint declarations, group definitions and constraint  violations. JAX-RS
implementations MUST provide a default exception mapper (see Section~\ref{exceptionmapper}) for
javax.validation.ValidationException according to the following rules:

\begin{enumerate}
    \item If the exception is of type javax.validation.ValidationException or any of its subclasses
    excluding javax.validation.ConstraintViolationException, then it is mapped to a response with status code 500
    (Internal Server Error).
    \item If the exception is an instance of javax.validation.ConstraintViolationException, then:
    \begin{enumerate}
        \item If the exception was thrown while validating a method return type, then it is mapped to a response with
        status code 500 (Internal Server Error) \footnote{The property path of a ConstraintViolation provides
        information about the location from which an exception originated. See Javadoc for more information.}.
        \item Otherwise, it is mapped to a response with status code 400 (Bad Request).
    \end{enumerate}
\end{enumerate}

In all cases, JAX-RS implementations SHOULD include a response entity describing the source of the error; however, the
exact content and format of this entity is beyond the scope of this specification. As described in
Section~\ref{exceptionmapper}, applications can provide their own exception mappers and, consequently, customize the
default mapper described above.

    \chapter{Asynchronous Processing}
\label{asynchronous_processing}

This chapter describes the asynchronous processing capabilities in JAX-RS. Asynchronous processing is supported both in
the Client API and in the Server API.

\section{Introduction}
\label{introduction_async}

Asynchronous processing is a technique that enables a better and more efficient use of processing threads. On the client
side, a thread that issues a request may also be responsible for updating a UI component; if that thread is blocked
waiting for a response, the user's perceived performance of the application will suffer. Similarly, on the server side,
a thread that is processing a request should avoid blocking while waiting for an external event to complete so that
other requests arriving to the server during that period of time can be attended\footnote{The maximum number of request
threads is typically set by the administrator; if that upper bound is reached, subsequent requests will be rejected.}.

\section{Server API}
\label{server_api}

\subsection{AsyncResponse}
\label{async_response}

Synchronous processing requires a resource method to produce a response upon returning control back to the JAX-RS
implementation. Asynchronous processing enables a resource method to inform the JAX-RS implementation that a response is
not readily available upon return but will be produced at a future time. This can be accomplished by first suspending
and later resuming the client connection on which the request was received.

Let us illustrate these concepts via an example:

\begin{lstlisting}[language=Java]
@Path("/async/longRunning")
public class MyResource {
    
    @GET
    public void longRunningOp(@Suspended final AsyncResponse ar) {
        executor.submit(
            new Runnable() {
                public void run() {
                    executeLongRunningOp();
                    ar.resume("Hello async world!");
                } 
            });
    } 
    ...
}
\end{lstlisting}

\textcolor{highlight green}{A resource method that elects to produce a response asynchronously must inject as a method parameter an instance of the
class \lstinline{AsyncResponse} using the special annotation \lstinline{@Suspended}}. In the example above, the method
\lstinline{longRunningOp} is called upon receiving a \lstinline{GET} request. Rather than producing a response
immediately, this method forks a (non-request) thread to execute a long running operation and returns immediately. Once
the execution of the long running operation is complete, the connection is resumed and the response returned by calling
\lstinline{resume} on the injected instance of \lstinline{AsyncResponse}.

For more information on executors, concurrency and thread management in a Java EE environment, the reader is referred to
JSR 236. For more information about executors in the JAX-RS Client API see Section \ref{executor_services}.

\subsubsection{Timeouts and Callbacks}
\label{timeouts_and_callbacks}

A timeout value can be specified when suspending a connection to avoid waiting for a response indefinitely. The default
unit is milliseconds, but any unit of type \lstinline{java.util.concurrent.TimeUnit} can be used. The following example
sets a timeout of 15 seconds and registers an instance of \lstinline{TimeoutHandler} in case the timeout is reached
before the connection is resumed.

\begin{lstlisting}[language=Java]
    @GET
    public void longRunningOp(@Suspended final AsyncResponse ar) {
        // Register handler and set timeout
        ar.setTimeoutHandler(new TimeoutHandler() {
            public void handleTimeout(AsyncResponse ar) {
                ar.resume(Response.status(SERVICE_UNAVAILABLE).entity(
                    "Operation timed out -- please try again").build());                    
                }
        });
        ar.setTimeout(15, SECONDS);
        
        // Execute long running operation in new thread
        executor.execute(
            new Runnable() {
                public void run() {
                    executeLongRunningOp();
                    ar.resume("Hello async world!");
                } 
            });
    }
\end{lstlisting}

JAX-RS implementations are REQUIRED to generate a \lstinline{ServiceUnavailableException}, a subclass of
WebApplicationException with its status set to 503, if the timeout value is reached and no timeout handler is
registered. The exception MUST be processed as described in section~\ref{method_exc}. If a registered timeout handler
resets the timeout value or resumes the connection and returns a response, JAX-RS implementations MUST NOT generate an
exception.

It is also possible to register callbacks on an instance of \lstinline{AsyncResponse} in order to listen for processing
completion (\lstinline{CompletionCallback}) and connection termination (\lstinline{ConnectionCallback}) events. See
Javadoc for \lstinline{AsyncResponse} for more information on how to register these callbacks. Note that support for
\lstinline{ConnectionCallback} is OPTIONAL.

\subsection{CompletionStage}
\label{completionstage}

An alternative approach to the injection of \lstinline{AsyncResponse} is for a resource method to return an instance of
\lstinline{CompletionStage<T>} as an indication to the underlying JAX-RS implementation that asynchronous processing is
enabled. The example from Section \ref{async_response} can be re-written using \lstinline{CompletionStage} as follows:

\begin{lstlisting}[language=Java]
	@Path("/async/longRunning")
	public class MyResource {
		
		@GET
		public CompletionStage<String> longRunningOp() {
			CompletableFuture<String> cs = new CompletableFuture<>();
			executor.submit(
			    new Runnable() {
				    public void run() {
					    executeLongRunningOp();
					    cs.complete("Hello async world!");
		  	        } 
	  	        });
	  	    return cs;
		}

		...
	}
\end{lstlisting}

In this example, a \lstinline{CompletableFuture} instance is created and returned in the resource method; the call to
method \lstinline{complete} on that instance is executed only after the long running operation terminates.

\section{EJB Resource Classes}
\label{async_ejbs}

As stated in Section~\ref{ejbs}, JAX-RS implementations in products that support EJB must also support the use of
stateless and singleton session beans as root resource classes. When an EJB method is annotated with
\lstinline{@Asynchronous}, the EJB container automatically allocates the necessary resources for its execution. Thus, in
this scenario, the use of an \lstinline{Executor} is unnecessary to generate an asynchronous response.

Consider the following example:

\begin{lstlisting}[language=Java]
@Stateless 
@Path("/")
class EJBResource {

    @GET @Asynchronous
    public void longRunningOp(@Suspended AsyncResponse ar) {
        executeLongRunningOp();
        ar.resume("Hello async world!");
    }
}
\end{lstlisting}

There is no explicit thread management needed in this case since that is under the control of the EJB container. Just
like the other examples in this chapter, the response is produced by calling \lstinline{resume} on the injected
\lstinline{AsyncResponse}. Hence, the return type of \lstinline{longRunningOp} is simply \lstinline{void}.

\section{Client API}
\label{client_api_async}

The fluent API supports asynchronous invocations as part of the invocation building process. By default, invocations are
synchronous but can be set to run asynchronously by calling the \lstinline{async} method and (optionally) registering an
instance of \lstinline{InvocationCallback} as shown next:

\begin{lstlisting}[language=Java]
Client client = ClientBuilder.newClient();
WebTarget target = client.target("http://example.org/customers/{id}");
target.resolveTemplate("id", 123).request().async().get(
    new InvocationCallback<Customer>() {
        @Override
        public void completed(Customer customer) {
            // Do something
        }

        @Override
        public void failed(Throwable throwable) {
            // Process error
        }
    });
\end{lstlisting}

Note that in this example, the call to \lstinline{get} after calling \lstinline{async} returns immediately without
blocking the caller's thread.

The response type is specified as a type parameter to \lstinline{InvocationCallback}. The method \lstinline{completed}
is called when the invocation completes successfully and a response is available; the method \lstinline{failed} is
called with an instance of \lstinline{Throwable} when the invocation fails.

All asynchronous invocations return an instance of \lstinline{Future<T>} here the type parameter \lstinline{T} matches
the type specified in \lstinline{InvocationCallback}. This instance can be used to monitor or cancel the asynchronous
invocation:

\begin{lstlisting}[language=Java]
Future<Customer> ff = target.resolveTemplate("id", 123).request().async()
    .get(new InvocationCallback<Customer>() {
        @Override
        public void completed(Customer customer) {
            // Do something
        }
        @Override
        public void failed(Throwable throwable) {
            // Process error
        }
    });

// After waiting for a while ...
if (!ff.isDone()) {
    ff.cancel(true);
} 
\end{lstlisting}

Even though it is recommended to pass an instance of \lstinline{InvocationCallback} when executing an asynchronous call,
it is not mandated. When omitted, the \lstinline{Future<T>} returned by the invocation can be used to gain access to the
response by calling the method \lstinline{Future.get}, which will return an instance of \lstinline{T} if the invocation
was successful or \lstinline{null} if the invocation failed.

    \chapter{Server-Sent Events}
\label{sse}

\section{Introduction}
\label{sse_introduction}

Server-sent events (SSE) is a specification originally introduced as part of HTML5 by the W3C, but is currently
maintained by the WHATWG. It provides a way to establish a one-way channel from a server to a client. The connection is
long running: it is re-used for multiple events sent from the server, yet it is still based on the HTTP protocol.
Clients request the opening of an SSE connection by using the special media type \lstinline{text/event-stream} in the
\lstinline{Accept} header.

Events are structured and contain several fields, namely, \lstinline{event}, \lstinline{data}, \lstinline{id},
\lstinline{retry} and \lstinline{comment}. SSE is a messaging protocol where the \lstinline{event} field corresponds to
a topic, and where the \lstinline{id} field can be used to validate event order and guarantee continuity. If a
connection is interrupted for any reason, the \lstinline{id} can be sent in a request header for a server to re-play
past events -- although this is an optional behavior that may not be supported by all implementations. Event payloads are conveyed in the \lstinline{data} field and must be in text format; \lstinline{retry} is used to control re-connects and finally \lstinline{comment} is a general purpose field that can also be used to keep connections alive.

\section{Client API}
\label{sse_client_api}

The JAX-RS client API for SSE was inspired by the corresponding JavaScript API in HTML5, but with changes that originate
from the use of a different language. The entry point to the client API is the type \lstinline{SseEventSource}, which
provides a fluent builder similarly to other classes in the JAX-RS API. An \lstinline{SseEventSource} is constructed
from a \lstinline{WebTarget} that is already configured with a resource location; \lstinline{SseEventSource} does not
duplicate any functionality in \lstinline{WebTarget} and only adds the necessary logic for SSE.

The following example shows how to open an SSE connection and read some messages for a little while:

\begin{lstlisting}[language=Java]
  WebTarget target = client.target("http://...");

  try (SseEventSource source = SseEventSource.target(target).build()) {
      source.register(System.out::println);
      source.open();


      Thread.sleep(500); // Consume events for just 500 ms

  } catch (InterruptedException e) {

      // falls through

  }
\end{lstlisting}

As seen in this example, an \lstinline{SseEventSource} implements \lstinline{AutoCloseable}. Before opening the source,
the client registers an event consumer that simply prints each event. Additional handlers for other life-cycle events
such as \lstinline{onComplete} and \lstinline{onError} are also supported, but for simplicity only \lstinline{onEvent}
is shown in the example above.

\section{Server API}
\label{sse_server_api}

The JAX-RS SSE server API is used to accept connections and send events to one or more clients. A resource method that
injects an \lstinline{SseEventSink} and produces the media type \lstinline{text/event-stream} is an SSE resource method.

The following example accepts SSE connections and uses an executor thread to send 3 events before closing the
connection:

\begin{lstlisting}[language=Java]
    @GET

    @Path("eventStream")

    @Produces(MediaType.SERVER_SENT_EVENTS)

    public void eventStream(@Context SseEventSink eventSink,
@Context Sse sse) {
        executor.execute(() -> {
            try (SseEventSink sink = eventSink) {


            eventSink.send(sse.newEvent("event1"));
            eventSink.send(sse.newEvent("event2"));
            eventSink.send(sse.newEvent("event3"));
        }
        });
    }
\end{lstlisting}

SSE resource methods follow a similar pattern to those for asynchronous processing
(see Section \ref{introduction_async}) in that the object representing the incoming connection, in this case
\lstinline{SseEventSink}, is injected into the resource method.

The example above also injects the \lstinline{Sse} type which provides factory methods for events and broadcasters. See
section \ref{sse_broadcasting} for more information about broadcasting. Note that, just like \lstinline{SseEventSource},
the interface \lstinline{SseEventSink} is also auto-closeable, hence the use of the try-with-resources statement
above.

Method \lstinline{send} on \lstinline{SseEventSink} returns a \lstinline{CompletionStage<?>} as a way to provide a
handle to the action of asynchronously sending a message to a client.

\section{Broadcasting}
\label{sse_broadcasting}

Applications may need to send events to multiple clients simultaneously. This action is called broadcasting in JAX-RS.
Multiple \lstinline{SseEventSink}'s can be registered on a single \lstinline{SseBroadcaster}.

A broadcaster can only be created by calling method \lstinline{newBroadcaster} on the injected \lstinline{Sse} instance.
The life-cycle and scope of an \lstinline{SseBroadcaster} is fully controlled by applications and not the JAX-RS
runtime. The following example shows the use of broadcasters, note the \lstinline{@Singleton} annotation on the resource
class:

\begin{lstlisting}[language=Java]
    @Path("/")

    @Singleton
    public class SseResource {

        @Context
        private Sse sse;

        private volatile SseBroadcaster sseBroadcaster;

        @PostConstruct
        public init() {
            this.sseBroadcaster = sse.newBroadcaster();
        }



        @GET
        @Path("register")
        @Produces(MediaType.SERVER_SENT_EVENTS)
        public void register(@Context SseEventSink eventSink) {
            eventSink.send(sse.newEvent("welcome!"));
            sseBroadcaster.register(eventSink);
        }



        @POST
        @Path("broadcast")
        @Consumes(MediaType.MULTIPART_FORM_DATA)
        public void broadcast(@FormParam("event") String event) {

            sseBroadcaster.broadcast(sse.newEvent(event));
        }

    }
\end{lstlisting}

The \lstinline{register} method on a broadcaster is used to add a new \lstinline{SseEventSink}; the
\lstinline{broadcast} method is used to send an SSE event to all registered consumers.

\section{Processing Pipeline}
\label{sse_pipeline}

Connections from SSE clients are represented by injectable instances of \lstinline{SseEventSink}. There are some
similarities between SSE and asynchronous processing (see Chapter \ref{asynchronous_processing}). Asynchronous responses
can be resumed at most once while an \lstinline{SseEventSink} can be used multiple times to stream individual events.

For compatibility purposes, implementations MUST initiate processing of an SSE response when either the first message is
sent or when the resource method returns, whichever happens first. The initial SSE response, which may only include the
HTTP headers, is processed using the standard JAX-RS pipeline as described in Appendix \ref{processing_pipeline}. Each
subsequent SSE event may include a different payload and thus require the use of a specific message body writer. Note
that since this use case differs slightly from the normal JAX-RS pipeline, implementations SHOULD NOT call entity
interceptors on each individual event \footnote{As a matter of fact, there is no API to bind entity interceptors to
individual SSE events.}.

\section{Environment}
\label{sse_environment}

The \lstinline{SseEventSource} class uses the existing JAX-RS mechanism based on \lstinline{RuntimeDelegate} to find an
implementation using the service name \lstinline{javax.ws.rs.sse.SseEventSource.Builder}. The majority of types in the
\lstinline{javax.ws.rs.sse} are thread safe; the reader is referred to the Javadoc for more information on thread
safety.

    \include{chapters/context}
    \chapter{Environment}
\label{environment}

The container-managed resources available to a JAX-RS root resource class or provider depend on the environment in which
it is deployed. Section \ref{contexttypes} describes the types of context available regardless of container. The
following sections describe the additional container-managed resources available to a JAX-RS root resource class or
provider deployed in a variety of environments.

\section{Servlet Container}
\label{servlet_container}

The \lstinline{@Context} annotation can be used to indicate a dependency on a Servlet-defined resource. A Servlet-based
implementation MUST support injection of the following Servlet-defined types: \lstinline{ServletConfig},
\lstinline{ServletContext}, \lstinline{HttpServletRequest} and \lstinline{HttpServletResponse}.

An injected \lstinline{HttpServletRequest} allows a resource method to stream the contents of a request entity. If the
resource method has a parameter whose value is derived from the request entity then the stream will have already been
consumed and an attempt to access it MAY result in an exception.

An injected \lstinline{HttpServletResponse} allows a resource method to commit the HTTP response prior to returning. An
implementation MUST check the committed status and only process the return value if the response is not yet committed.

Servlet filters may trigger consumption of a request body by accessing request parameters. In a servlet container the
@FormParam annotation and the standard entity provider for \lstinline{application/x--www-form-urlencoded} MUST obtain
their values from the servlet request parameters if the request body has already been consumed. Servlet APIs do not
differentiate between parameters in the URI and body of a request so URI-based query parameters may be included in the
entity parameter.

\section{Integration with Java EE Technologies}
\label{javaee}

This section describes the additional requirements that apply to a JAX-RS implementation when combined in a product that
supports the following specifications.

\subsection{Servlets}
\label{servlets}

In a product that also supports the Servlet specification, implementations MUST support JAX-RS applications that are
packaged as a Web application. See Section \ref{servlet} for more information Web application packaging.

It is RECOMMENDED for a JAX-RS implementation to provide asynchronous processing support, as defined in Chapter
\ref{asynchronous_processing}, by enabling asynchronous processing (i.e., \lstinline{asyncSupported=true}) in the
underlying Servlet 3 container. It is OPTIONAL for a JAX-RS implementation to support asynchronous processing when
running on a Servlet container whose version is prior to 3.

As explained in Section \ref{servlet_container}, injection of Servlet-defined types is possible using the
\lstinline{@Context} annotation. Additionally, web application's \lstinline{<context-param>} and servlet's
\lstinline{<init-param>} can be used to define application properties passed to server-side features or injected into
server-side JAX-RS components. See Javadoc for \lstinline{Application.getProperties} for more information.

\subsection{Managed Beans}
\label{managed_beans}

In a product that supports Managed Beans, implementations MUST support the use of Managed Beans as root resource
classes, providers and \lstinline{Application} subclasses.

For example, a bean that uses a managed-bean interceptor can be defined as a JAX-RS resource as follows:

\begin{lstlisting}[language=Java]
    @ManagedBean
    @Path("/managedbean")
    public class ManagedBeanResource {

    public static class MyInterceptor {
    @AroundInvoke
    public String around(InvocationContext ctx) throws Exception {
    System.out.println("around() called");
    return (String) ctx.proceed();
    }
    }

    @GET
    @Produces("text/plain")
    @Interceptors(MyInterceptor.class)
    public String getIt() {
    return "Hi managedbean!";
    }
    }
\end{lstlisting}

The example above uses a managed-bean interceptor to intercept calls to the resource method \lstinline{getIt}. See
Section \ref{additional_reqs} for additional requirements on Managed Beans.

\subsection{Context and Dependency Injection (CDI)}
\label{cdi}
In a product that supports CDI, implementations MUST support the use of CDI-style Beans as root resource classes,
providers and \lstinline{Application} subclasses. Providers and \lstinline{Application} subclasses MUST be singletons or
use application scope.

For example, assuming CDI is enabled via the inclusion of a \lstinline{beans.xml} file, a CDI-style bean that can be
defined as a JAX-RS resource as follows:

\begin{lstlisting}[language=Java]
    @Path("/cdibean")
    public class CdiBeanResource {

    @Inject MyOtherCdiBean bean; // CDI injected bean

    @GET
    @Produces("text/plain")
    public String getIt() {
    return bean.getIt();
    }
    }
\end{lstlisting}

The example above takes advantage of the type-safe dependency injection provided in CDI by using another bean, of type
\lstinline{MyOtherCdiBean}, in order to return a resource representation. See Section \ref{additional_reqs} for
additional requirements on CDI-style Beans.

\subsection{Enterprise Java Beans (EJBs)}
\label{ejbs}

In a product that supports EJBs, an implementation MUST support the use of stateless and singleton session beans as root
resource classes, providers and \lstinline{Application} subclasses.
JAX-RS annotations can be applied to methods in an EJB's local interface or directly to methods in a no-interface EJB.
Resource class annotations (like \lstinline{@Path}) MUST be applied to an EJB's class directly following the annotation
inheritance rules defined in Section \ref{annotationinheritance}.

For example, a stateless EJB that implements a local interface can be defined as a JAX-RS resource class as follows:

\begin{lstlisting}[language=Java]
    @Local
    public interface LocalEjb {

    @GET
    @Produces("text/plain")
    public String getIt();
    }

    @Stateless
    @Path("/stateless")
    public class StatelessEjbResource implements LocalEjb {

    @Override
    public String getIt() {
    return "Hi stateless!";
    }
    }
\end{lstlisting}

JAX-RS implementations are REQUIRED to discover EJBs by inspecting annotations on classes and local interfaces; they are
not REQUIRED to read EJB deployment descriptors (ejb-jar.xml). Therefore, any information in an EJB deployment
descriptor for the purpose of overriding EJB annotations or providing additional meta-data will likely result in a
non-portable JAX-RS application.

If an \lstinline{Exception\-Mapper} for a \lstinline{EJBException} or subclass is not included with an application then
exceptions thrown by an EJB resource class or provider method MUST be unwrapped and processed as described in Section
\ref{method_exc}.

See Section \ref{async_ejbs} for more information on asynchronous EJB methods and Section \ref{additional_reqs} for
additional requirements on EJBs.

\subsection{Bean Validation}
\label{bv_support}

In a product that supports the Bean Validation specification \cite{bv11}, implementations MUST support resource
validation using constraint annotations as described in Chapter \ref{validation}. Otherwise, support for resource
validation is OPTIONAL.

\subsection{Java API for JSON Processing}
\label{jsonp}

In a product that supports the Java API for JSON Processing (JSON-P) \cite{jsonp}, implementations MUST support entity
providers for \lstinline{JsonValue} and all of its sub-types: \lstinline{JsonStructure}, \lstinline{JsonObject},
\lstinline{JsonArray}, \lstinline{JsonString} and \lstinline{JsonNumber}.

Note that other types from the JSON-P API such as \lstinline{JsonParser}, \lstinline{JsonGenerator},
\lstinline{JsonReader} and \lstinline{JsonWriter} can also be integrated into JAX-RS applications using the entity
providers for \lstinline{InputStream} and \lstinline{StreamingOutput}.


\subsection{Java API for JSON Binding}
\label{jsonb}
In a product that supports the Java API for JSON Binding (JSON-B) \cite{jsonb},
implementations MUST support entity providers for all Java types supported by JSON-B in combination with the following
media types: \lstinline{application/json}, \lstinline{text/json} as well as any other media types matching
\lstinline{*/json} or \lstinline{*/*+json}.

Note that if JSON-B and JSON-P are both supported in the same environment, entity providers for JSON-B take precedence
over those for JSON-P for all types except \lstinline{JsonValue} and its sub-types.


\subsection{Additional Requirements}
\label{additional_reqs}

The following additional requirements apply when using Managed Beans, CDI-style Beans or EJBs as resource classes,
providers or \lstinline{Application} subclasses:

\begin{itemize}
    \item Field and property injection of JAX-RS resources MUST be performed prior to the container invoking any
    \lstinline{@PostConstruct} annotated method.
    \item Support for constructor injection of JAX-RS resources is OPTIONAL. Portable applications MUST instead use
    fields or bean properties in conjunction with a \lstinline{@PostConstruct} annotated method. Implementations SHOULD
    warn users about use of non-portable constructor injection.
    \item Implementations MUST NOT require use of \lstinline{@Inject} or \lstinline{@Resource} to trigger injection of
    JAX-RS annotated fields or properties. Implementations MAY support such usage but SHOULD warn users about
    non-portability.
\end{itemize}

\section{Other}

Other container technologies MAY specify their own set of injectable resources but MUST, at a minimum, support access to
the types of context listed in Section \ref{contexttypes}.

    \include{chapters/delegate}

    \appendix
    \renewcommand{\chaptermark}[1]{
      \markboth{\appendixname
      \ \thechapter.\ #1}{}}

    \include{chapters/diagrams}
\end{document}
