\chapter{Introduction}

This specification defines a set of Java APIs for the development of Web services built according to the
Representational State Transfer (REST) architectural style.

\section{Status}
\label{status}

This is the final release of version 2.1. The issue tracking system for this release can be found at
\url{https://github.com/jJAX-RS/api/issues}

The corresponding Javadocs can be found online at \url{https://jJAX-RS.github.io/apidocs/2.1}

\textcolor{highlight green}{The reference implementation can be obtained from \url{https://jersey.github.io}}
\footnote{This is one advantage of Jersey over Spring, i.e. portability}

\section{Goals}

The following are the goals of the API:

\begin{description}
    \item[POJO-based] The API will provide a set of annotations and associated classes/interfaces that may be used with
    POJOs in order to expose them as Web resources. The specification will define object lifecycle and scope.
    \item[HTTP-centric] The specification will assume HTTP is the underlying network protocol and will provide a clear
    mapping between HTTP and URI elements and the corresponding API classes and annotations. The API will provide high
    level support for common HTTP usage patterns and will be sufficiently flexible to support a variety of HTTP
    applications including WebDAV and the Atom Publishing Protocol.
    \item[Format independence] The API will be applicable to a wide variety of HTTP entity body content types. It will
    provide the necessary pluggability to allow additional types to be added by an application in a standard manner.
    \item[Container independence] Artifacts using the API will be deployable in a variety of Web-tier containers. The
    specification will define how artifacts are deployed in a Servlet container and as a JAX-WS Provider.
    \item[Inclusion in Java EE] The specification will define the environment for a Web resource class hosted in a Java
    EE container and will specify how to use Java EE features and components within a Web resource class.
\end{description}

\section{Non-Goals}
\label{non_goals}

The following are non-goals:

\begin{description}
    \item[Support for Java versions prior to Java SE 8] The API will make extensive use of annotations and lambda
    expressions that require Java SE 8 or later.
    \item[Description, registration and discovery] The specification will neither define nor require any service
    description, registration or discovery capability.
    \item[HTTP Stack] The specification will not define a new HTTP stack. HTTP protocol support is provided by a
    container that hosts artifacts developed using the API.
    \item[Data model/format classes] The API will not define classes that support manipulation of entity body content,
    rather it will provide pluggability to allow such classes to be used by artifacts developed using the API.
\end{description}

\section{Terminology}
\label{terminology}

\begin{description}
    \item[Resource class] A Java class that uses JAX-RS annotations to implement a corresponding Web resource, see
    Chapter \ref{resources}.
    \item[Root resource class] A resource class annotated with \lstinline{@Path}. Root resource classes provide the
    roots of the resource class tree and provide access to sub-resources, see Chapter \ref{resources}.
    \item[Request method designator] A runtime annotation annotated with \lstinline{@HttpMethod}. Used to identify the
    HTTP request method to be handled by a resource method.
    \item[Resource method] A method of a resource class annotated with a request method designator that is used to
    handle requests on the corresponding resource, see Section \ref{resource_method}.
    \item[Sub-resource locator] A method of a resource class that is used to locate sub-resources of the corresponding
    resource, see Section \ref{sub_resources}.
    \item[Sub-resource method] A method of a resource class that is used to handle requests on a sub-resource of the
    corresponding resource, see Section \ref{sub_resources}.
    \item[Provider] An implementation of a JAX-RS extension interface. Providers extend the capabilities of a JAX-RS
    runtime and are described in Chapter \ref{providers}.
    \item[Filter] A provider used for filtering requests and responses.
    \item[Entity Interceptor] A provider used for intercepting calls to message body readers and writers.
    \item[Invocation] A Client API object that can be configured to issue an HTTP request.
    \item[WebTarget] The recipient of an Invocation, identified by a URI.
    \item[Link] A URI with additional meta-data such as a media type, a relation, a title, etc.
\end{description}
