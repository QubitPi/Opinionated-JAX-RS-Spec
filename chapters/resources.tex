\chapter{Resources}
\label{resources}

Using JAX-RS, \textcolor{highlight green}{a Web resource is implemented as a resource class and requests are handled by
resource methods}\footnote{Endpoints should exists in a package called "resource" for readability}.

\section{Resource Classes}

\textcolor{highlight green}{A resource class is a Java class that uses JAX-RS annotations to implement a corresponding
Web resource. Resource classes are POJOs that have at least one method annotated with \lstinline{@Path} or a request method
designator}.

\subsection{Lifecycle and Environment}

\textcolor{highlight green}{By default a new resource class instance is created for each request} to that resource.
First the constructor (see Section \ref{resource_class_constructor}) is called, then any requested dependencies are
injected (see Section \ref{resource_field}), then the appropriate method (see Section \ref{resource_method}) is invoked
and finally the object is made available for garbage collection.

\subsection{Constructors}
\label{resource_class_constructor}

Root resource classes are instantiated by the JAX-RS runtime and MUST have a public constructor for which the JAX-RS
runtime can provide all parameter values. Note that a zero argument constructor is permissible under this rule.

A public constructor MAY include parameters annotated with one of the following: \lstinline{@Context},
\lstinline{@HeaderParam}, \lstinline{@CookieParam}, \lstinline{@MatrixParam}, \lstinline{@QueryParam} or
\lstinline{@PathParam}. However, depending on the resource class lifecycle and concurrency, per-request information may
not make sense in a constructor. If more than one public constructor is suitable then an implementation MUST use the one
with the most parameters. Choosing amongst suitable constructors with the same number of parameters is implementation
specific, implementations SHOULD generate a warning about such ambiguity.

Non-root resource classes are instantiated by an application and do not require the above-described public constructor.

\section{Fields and Bean Properties}
\label{resource_field}

When a resource class is instantiated, the values of fields and bean properties annotated with one the following
annotations are set according to the semantics of the annotation:

\begin{description}
    \item[\lstinline{@MatrixParam}] Extracts the value of a URI matrix parameter.
    \item[\lstinline{@QueryParam}] Extracts the value of a URI query parameter.
    \item[\lstinline{@PathParam}] Extracts the value of a URI template parameter.
    \item[\lstinline{@CookieParam}] Extracts the value of a cookie.
    \item[\lstinline{@HeaderParam}] Extracts the value of a header.
    \item[\lstinline{@Context}] Injects an instance of a supported resource, see chapters \ref{context} and
    \ref{environment} for more details.
\end{description}

Because injection occurs at object creation time, use of these annotations (with the exception of \lstinline{@Context})
on resource class fields and bean properties is only supported for the default per-request resource class lifecycle. An
implementation SHOULD warn if resource classes with other lifecycles use these annotations on resource class fields or
bean properties.

A JAX-RS implementation is only required to set the annotated field and bean property values of instances created by its
runtime. Objects returned by sub-resource locators (see Section \ref{sub_resources}) are expected to be initialized by
their creator.

Valid parameter types for each of the above annotations are listed in the corresponding Javadoc, however in general
(excluding \lstinline{@Context}) the following types are supported:

\begin{enumerate}
    \item\label{paramconverter} Types for which a ParamConverter is available via a registered ParamConverterProvider.
    See Javadoc for these classes for more information.
    \item Primitive types.
    \item\label{stringctor} Types that have a constructor that accepts a single \lstinline{String} argument.
    \item\label{valueofmethod} Types that have a static method named \lstinline{valueOf} or \lstinline{fromString} with
    a single \lstinline{String} argument that return an instance of the type. If both methods are present then
    \lstinline{valueOf} MUST be used unless the type is an enum in which case \lstinline{fromString} MUST be
    used\footnote{Due to limitations of the built-in \lstinline{valueOf} method that is part of all Java enumerations, a
    \lstinline{fromString} method is often defined by the enum writers. Consequently, the \lstinline{fromString} method
    is preferred when available.}.
    \item \lstinline{List<T>}, \lstinline{Set<T>}, or \lstinline{SortedSet<T>}, where \lstinline{T} satisfies
    \ref{paramconverter}, \ref{stringctor} or \ref{valueofmethod} above.
\end{enumerate}

The \lstinline{DefaultValue} annotation may be used to supply a default value for some of the above, see the Javadoc for
\lstinline{DefaultValue} for usage details and rules for generating a value in the absence of this annotation and the
requested data. The \lstinline{Encoded} annotation may be used to disable automatic URI decoding for
\lstinline{@MatrixParam}, \lstinline{@QueryParam}, and \lstinline{@PathParam} annotated fields and properties.

A WebApplicationException thrown during construction of field or property values using any of the 5 steps listed above
is processed directly as described in Section \ref{method_exc}. Other exceptions thrown during construction of field or
property values using any of the 5 steps listed above are treated as client errors: if the field or property is
annotated with \lstinline{@MatrixParam}, \lstinline{@QueryParam}, and \lstinline{@PathParam} then an implementation MUST
generate an instance of \lstinline{NotFoundException} (404 status) that wraps the thrown exception and no entity; if the
field or property is annotated with \lstinline{@HeaderParam} or \lstinline{@CookieParam} then an implementation MUST
generate an instance of \lstinline{BadRequestException} (400 status) that wraps the thrown exception and no entity.
Exceptions MUST be processed as described in Section \ref{method_exc}.

\section{Resource Methods}
\label{resource_method}

Resource methods are methods of a resource class annotated with a request method designator. They are used to handle
requests and MUST conform to certain restrictions described in this section.

A request method designator is a runtime annotation that is annotated with the \lstinline{@HttpMethod} annotation.
JAX-RS defines a set of request method designators for the common HTTP methods: \lstinline{@GET}, \lstinline{@POST},
\lstinline{@PUT}, \lstinline{@DELETE}, \lstinline{@PATCH}, \lstinline{@HEAD} and \lstinline{@OPTIONS}. Users may define
their own custom request method designators including alternate designators for the common HTTP methods.

\subsection{Visibility}
\label{visibility}

Only \lstinline{public} methods may be exposed as resource methods. An implementation SHOULD warn users if a non-public
method carries a method designator or \lstinline{@Path} annotation.

\subsection{Parameters}
\label{resource_method_params}

When a resource method is invoked, parameters annotated with \lstinline{@FormParam} or one of the annotations listed in
Section \ref{resource_field} are mapped from the request according to the semantics of the annotation. Similar to fields
and bean properties:

\begin{itemize}
    \item The DefaultValue annotation may be used to supply a default value for parameters
    \item The Encoded annotation may be used to disable automatic URI decoding of parameter values
    \item Exceptions thrown during construction of parameter values are treated the same as exceptions thrown during
    construction of field or bean property values, see Section \ref{resource_field}. Exceptions thrown during
    construction of \lstinline{@FormParam} annotated parameter values are treated the same as if the parameter were
    annotated with \lstinline{@HeaderParam}.
\end{itemize}

\subsubsection{Entity Parameters}
\label{entity_parameters}

The value of a parameter not annotated with \lstinline{@FormParam} or any of the annotations listed in in Section
\ref{resource_field}, called the entity parameter, is mapped from the request entity body. Conversion between an entity
body and a Java type is the responsibility of an entity provider, see Section \ref{entity_providers}. Resource methods
MUST have at most one entity parameter.

\subsection{Return Type}
\label{resource_method_return}

\textcolor{highlight green}{Resource methods MAY return \lstinline{void}, Response, GenericEntity, or another Java
type}, these return types are mapped to a response entity body as follows:

\begin{description}
    \item[\lstinline{void}] Results in an empty entity body with a 204 status code.
    \item[Response] Results in an entity body mapped from the entity property of the Response with the status code
    specified by the status property of the Response. A \lstinline{null} return value results in a 204 status code.
    If the status property of the Response is not set: a 200 status code is used for a non-null entity property and a
    204 status code is used if the entity property is \lstinline{null}.
    \item[GenericEntity] Results in an entity body mapped from the \lstinline{Entity} property of the GenericEntity. If
    the return value is not \lstinline{null} a 200 status code is used, a \lstinline{null} return value results in a 204
    status code.
    \item[Other] Results in an entity body mapped from the class of the returned instance or of its type parameter
    \lstinline{T} if the return type is \lstinline{CompletionStage<T>} (see Section \ref{completionstage}); if the class
    is an anonymous inner class, its superclass is used instead. If the return value is not \lstinline{null} a 200
    status code is used, a \lstinline{null} return value results in a 204 status code.
\end{description}

Methods that need to provide additional metadata with a response should return an instance of Response, the Response
\lstinline{Builder} class provides a convenient way to create a Response instance using a builder pattern.

\subsection{Exceptions}
\label{method_exc}

A resource method, sub-resource method or sub-resource locator may throw any checked or unchecked exception. An
implementation MUST catch all exceptions and process them in the following order:

\begin{enumerate}
    \item Instances of WebApplicationException and its subclasses MUST be mapped to a response as follows. If the
    \lstinline{response} property of the exception does not contain an entity and an exception mapping provider (see
    Section \ref{exceptionmapper}) is available for WebApplicationException or the corresponding subclass, an
    implementation MUST use the provider to create a new Response instance, otherwise the \lstinline{response} property
    is used directly. The resulting Response instance is then processed according to Section
    \ref{resource_method_return}.
    \item If an exception mapping provider (see Section \ref{exceptionmapper}) is available for the exception or one of
    its superclasses, an implementation MUST use the provider whose generic type is the nearest superclass of the
    exception to create a Response instance that is then processed according to Section \ref{resource_method_return}.
    If the exception mapping provider throws an exception while creating a Response then return a server error (status
    code 500) response to the client.
    \item\label{runtimeexc} Unchecked exceptions and errors that have not been mapped MUST be re-thrown and allowed to
    propagate to the underlying container.
    \item\label{checkedexc} Checked exceptions and throwables that have not been mapped and cannot be thrown directly
    MUST be wrapped in a container-specific exception that is then thrown and allowed to propagate to the underlying
    container. Servlet-based implementations MUST use \lstinline{Servlet\-Exception} as the wrapper. JAX-WS
    \lstinline{Provider}-based implementations MUST use \lstinline{Web\-Service\-Exception} as the wrapper.
\end{enumerate}

Note: Items \ref{runtimeexc} and \ref{checkedexc} allow existing container facilities (e.g. a Servlet filter or error
pages) to be used to handle the error if desired.

\subsection{HEAD and OPTIONS}
\label{head_and_options}

\lstinline{HEAD} and \lstinline{OPTIONS} requests receive additional automated support. On receipt of a \lstinline{HEAD}
request an implementation MUST either:

\begin{enumerate}
    \item Call a method annotated with a request method designator for \lstinline{HEAD} or, if none present,
    \item\label{get_not_head} Call a method annotated with a request method designator for \lstinline{GET} and discard
    any returned entity.
\end{enumerate}

\textcolor{highlight green}{Note that option \ref{get_not_head} may result in reduced performance where entity creation is significant.}

On receipt of an \lstinline{OPTIONS} request an implementation MUST either:

\begin{enumerate}
    \item Call a method annotated with a request method designator for \lstinline{OPTIONS} or, if none present,
    \item Generate an automatic response using the metadata provided by the JAX-RS annotations on the matching class and
    its methods.
\end{enumerate}

\section{URI Templates}
\label{uritemplates}

\textcolor{highlight green}{A root resource class is anchored in URI space using the \lstinline{@Path} annotation. The
value of the annotation is a relative URI path template whose base URI is provided by the combination of the deployment
context and the application path (see the \lstinline{@ApplicationPath} annotation)}.

A URI path template is a string with zero or more embedded parameters that, when values are substituted for all the
parameters, is a valid URI path. The Javadoc for the \lstinline{@Path} annotation describes their syntax. E.g.:

\begin{lstlisting}[language=Java]
    @Path("widgets/{id}")
    public class Widget {
        ...
    }
\end{lstlisting}

In the above example the \lstinline{Widget} resource class is identified by the relative URI path
\lstinline{widgets/xxx} where \lstinline{xxx} is the value of the \lstinline{id} parameter.

Note: Because \lq\{\rq and \lq\}\rq\ are not part of either the reserved or unreserved productions of URI they will not
appear in a valid URI.

The value of the annotation is automatically encoded, e.g. the following two lines are equivalent:

\begin{lstlisting}[language=Java]
    @Path("widget list/{id}")
    @Path("widget%20list/{id}")
\end{lstlisting}

Template parameters can optionally specify the regular expression used to match their values. The default value matches
any text and terminates at the end of a path segment but other values can be used to alter this behavior, e.g.:

\begin{lstlisting}[language=Java]
    @Path("widgets/{path:.+}")
    public class Widget {
        ...
    }
\end{lstlisting}

In the above example the \lstinline{Widget} resource class will be matched for any request whose path starts with
\lstinline{widgets} and contains at least one more path segment; the value of the \lstinline{path} parameter will be the
request path following \lstinline{widgets}. E.g. given the request path \lstinline{widgets/small/a} the value of
\lstinline{path} would be \lstinline{small/a}.

The value of a URI path parameter is available for injection via \lstinline{@PathParam} on a field, property or method
parameter. Note that if a URI template is used on a method, a path parameter injected in a field or property may not be
available (set to \lstinline{null}). The following example illustrates this scenario:

\begin{lstlisting}[language=Java]
    @Path("widgets")
    public class WidgetsResource {

        @PathParam("id") String id;

        @GET
        public WidgetList getWidgets() {
        ... // id is null here
        }

        @GET
        @Path("{id}")
        public Widget findWidget() {
        return new WidgetResource(id);
        }
    }
\end{lstlisting}

\subsection{Sub Resources}
\label{sub_resources}

\textcolor{highlight green}{Methods of a resource class that are annotated with \lstinline{@Path} are either
sub-resource methods or sub-resource locators}. Sub-resource methods handle a HTTP request directly whilst sub-resource
locators return an object or class that will handle a HTTP request. The presence or absence of a request method
designator (e.g. \lstinline{@GET}) differentiates between the two:

\begin{description}
    \item[Present] Such methods, known as {\em sub-resource methods}, are treated like a normal resource method (see
    Section \ref{resource_method}) except the method is only invoked for request URIs that match a URI template created
    by concatenating the URI template of the resource class with the URI template of the method\footnote{If the resource
    class URI template does not end with a \lq/\rq\ character then one is added during the concatenation.}.

    \item[Absent] Such methods, known as {\em sub-resource locators}, are used to dynamically resolve the object that
    will handle the request. Sub-resource locators can return objects or classes; if a class is returned then an object
    is obtained by the implementation using a suitable constructor as described in Section
    \ref{resource_class_constructor}. In either case, the resulting object is used to handle the request or to further
    resolve the object that will handle the request, see \ref{mapping_requests_to_java_methods} for further details.

    When an object is returned, implementations MUST dynamically determine its class rather than relying on the static
    sub-resource locator return type, since the returned instance may be a subclass of the declared type with
    potentially different annotations, see Section \ref{annotationinheritance} for rules on annotation inheritance.
    Sub-resource locators may have all the same parameters as a normal resource method (see Section
    \ref{resource_method}) except that they MUST NOT have an entity parameter.
\end{description}

The following example illustrates the difference:

\begin{lstlisting}[language=Java]
    @Path("widgets")
    public class WidgetsResource {

        @GET
        @Path("offers")
        public WidgetList getDiscounted() {...}

        @Path("{id}")
        public WidgetResource findWidget(@PathParam("id") String id) {
            return new WidgetResource(id);
        }
    }

    public class WidgetResource {

        public WidgetResource(String id) {...}

        @GET
        public Widget getDetails() {...}
    }
\end{lstlisting}

In the above a \lstinline{GET} request for the \lstinline{widgets/offers} resource is handled directly by the
\lstinline{getDiscounted} sub-resource method of the resource class
\lstinline{WidgetsResource} whereas a \lstinline{GET} request for \lstinline{widgets/xxx} is handled by the
\lstinline{getDetails} method of the \lstinline{WidgetResource} resource class.

Note: A set of sub-resource methods annotated with the same URI template value are functionally equivalent to a
similarly annotated sub-resource locator that returns an instance of a resource class with the same set of resource
methods.

\section{Declaring Media Type Capabilities}
\label{declaring_method_capabilities}

\textcolor{highlight green}{Application classes can declare the supported request and response media types using the
\lstinline{@Consumes} and \lstinline{@Produces} annotations respectively}. These annotations MAY be applied to a
resource method, a resource class, or to an entity provider (see Section \ref{declaring_provider_capabilities}). Use of
these annotations on a resource method overrides any on the resource class or on an entity provider for a method
argument or return type. In the absence of either of these annotations, support for any media type (\lq\lq*/*\rq\rq) is
assumed.

The following example illustrates the use of these annotations:

\begin{lstlisting}[language=Java]

    @Path("widgets")
    @Produces("application/widgets+xml")
    public class WidgetsResource {

        @GET
        public Widgets getAsXML() {...}

        @GET
        @Produces("text/html")
        public String getAsHtml() {...}

        @POST
        @Consumes("application/widgets+xml")
        public void addWidget(Widget widget) {...}
    }

    @Provider
    @Produces("application/widgets+xml")
    public class WidgetsProvider implements MessageBodyWriter<Widgets> {...}

    @Provider
    @Consumes("application/widgets+xml")
    public class WidgetProvider implements MessageBodyReader<Widget> {...}
\end{lstlisting}

In the above:
\begin{itemize}
    \item The \lstinline{getAsXML} resource method will be called for \lstinline{GET} requests that specify a response
    media type of \lstinline{application/widgets+xml}. It returns a \lstinline{Widgets} instance that will be mapped to
    that format using the \lstinline{WidgetsProvider} class (see Section \ref{entity_providers} for more information on
    MessageBodyWriter).
    \item The \lstinline{getAsHtml} resource method will be called for \lstinline{GET} requests that specify a response
    media type of \lstinline{text/html}. It returns a \lstinline{String} containing \lstinline{text/html} that will be
    written using the default implementation of \lstinline{MessageBodyWriter<String>}.
    \item The \lstinline{addWidget} resource method will be called for \lstinline{POST} requests that contain an entity
    of the media type \lstinline{application/widgets+xml}. The value of the \lstinline{widget} parameter will be mapped
    from the request entity using the \lstinline{WidgetProvider} class (see Section \ref{entity_providers} for more
    information on MessageBodyReader).
\end{itemize}

An implementation MUST NOT invoke a method whose effective value of \lstinline{@Produces} does not match the request
\lstinline{Accept} header. An implementation MUST NOT invoke a method whose effective value of \lstinline{@Consumes}
does not match the request \lstinline{Content-Type} header.

When accepting multiple media types, clients may indicate preferences by using a relative quality factor known as the q
parameter. The value of the q parameter, or q-value, is used to sort the set of accepted types. For example, a client
may indicate preference for \lstinline{application/widgets+xml} with a relative quality factor of 1 and for
\lstinline{application/xml} with a relative quality factor of 0.8. Q-values range from 0 (undesirable) to 1
(highly desirable), with 1 used as default when omitted. A \lstinline{GET} request matched to the
\lstinline{WidgetsResource} class with an accept header of \lstinline{text/html; q=1, application/widgets+xml; q=0.8}
will result in a call to method \lstinline{getAsHtml} instead of \lstinline{getAsXML} based on the value of q.

A server can also indicate media type preference using the qs parameter; server preference is only examined when
multiple media types are accepted by a client {\em with the same q-value}. Consider the following example:

\begin{lstlisting}[language=Java]
    @Path("widgets2")
    public class WidgetsResource2 {

        @GET
        @Produces("application/xml", "application/json")
        public Widgets getWidget() {...}
    }
\end{lstlisting}

Suppose a client issues a \lstinline{GET} request with an accept header of \lstinline{application/*; q=0.5, text/html}.
Based on this request, the server determines that both \lstinline{application/xml} and \lstinline{application/json} are
equally preferred by the client with a q-value of 0.5. By specifying a server relative quality factor as part of the
\lstinline{@Produces} annotation, it is possible to control which response media type to select:

\begin{lstlisting}[language=Java]
    @Path("widgets2")
    public class WidgetsResource2 {

        @GET
        @Produces("application/xml; qs=1", "application/json; qs=0.75")
        public Widgets getWidget() {...}
    }
\end{lstlisting}

With the updated value for \lstinline{@Produces} in this example, and in response to a \lstinline{GET} request with an
accept header that includes \lstinline{application/*; q=0.5}, JAX-RS implementations are REQUIRED to select the media
type \lstinline{application/xml} given its higher qs-value. Note that qs-values, just like q-values, are relative and as
such are only comparable to other qs-values within the same \lstinline{@Produces} annotation instance. For more
information see Section~\ref{determine_response_type}.

\section{Annotation Inheritance}
\label{annotationinheritance}

\textcolor{highlight green}{JAX-RS annotations may be used on the methods and method parameters of a super-class or an
implemented interface. Such annotations are inherited by a corresponding sub-class or implementation class method
provided that the method and its parameters do not have any JAX-RS annotations of their own. Annotations on a
super-class take precedence over those on an implemented interface. The precedence over conflicting annotations defined
in multiple implemented interfaces is implementation specific. Note that inheritance of class or interface annotations
is not supported.}

If a subclass or implementation method has any JAX-RS annotations then all of the annotations on the superclass or
interface method are ignored. E.g.:

\begin{lstlisting}[language=Java]
    public interface ReadOnlyAtomFeed {

        @GET
        @Produces("application/atom+xml")
        Feed getFeed();
    }

    @Path("feed")
    public class ActivityLog implements ReadOnlyAtomFeed {

        public Feed getFeed() {...}
    }
\end{lstlisting}

In the above, \lstinline{ActivityLog.getFeed} inherits the \lstinline{@GET} and \lstinline{@Produces} annotations from
the interface. Conversely:

\begin{lstlisting}[language=Java]
    @Path("feed")
    public class ActivityLog implements ReadOnlyAtomFeed {

        @Produces("application/atom+xml")
        public Feed getFeed() {...}
    }
\end{lstlisting}

In the above, the \lstinline{@GET} annotation on \lstinline{ReadOnlyAtomFeed.getFeed} is not inherited by
\lstinline{ActivityLog.getFeed} and it would require its own request method designator since it redefines the
\lstinline{@Produces} annotation.

For consistency with other Java EE specifications, it is recommended to always repeat annotations instead of relying on
annotation inheritance.

\section{Matching Requests to Resource Methods}
\label{mapping_requests_to_java_methods}

This section describes how a request is matched to a resource class and method. Implementations are not required to use the algorithm as written but MUST produce results equivalent to those produced by the algorithm.

\subsection{Request Preprocessing}
\label{reqpreproc}

Prior to matching, request URIs are normalized\footnote{Note: some containers might perform this functionality prior to
passing the request to an implementation.} by following the rules for case, path segment, and percent encoding
normalization described in section 6.2.2 of RFC 3986. The normalized request URI MUST be reflected in the URIs obtained
from an injected \lstinline{UriInfo}.

\subsection{Request Matching}
\label{request_matching}

A request is matched to the corresponding resource method or sub-resource method by comparing the
\textcolor{highlight green}{normalized} request URI (see Section \ref{reqpreproc}), the media type of any request
entity, and the requested response entity format to the metadata annotations on the resource classes and their methods.
If no matching resource method or sub-resource method can be found then an appropriate error response is returned. All
exceptions reported by this algorithm MUST be processed as described in Section \ref{method_exc}.

Matching of requests to resource methods proceeds in three stages as follows:

\begin{enumerate}
    \item Identify a set of candidate root resource classes matching the request:
    \item \label{find_object} Obtain a set of candidate resource methods for the request
    \item \label{find_method} Identify the method that will handle the request:
\end{enumerate}
