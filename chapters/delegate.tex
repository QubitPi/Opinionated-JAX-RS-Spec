\chapter{Runtime Delegate}
\label{runtimedelegate}

RuntimeDelegate is an abstract factory class that provides various methods for the creation of objects that implement
JAX-RS APIs. These methods are designed for use by other JAX-RS API classes and are not intended to be called directly
by applications. RuntimeDelegate allows the standard JAX-RS API classes to use different JAX-RS implementations without
any code changes.

An implementation of JAX-RS MUST provide a concrete subclass of RuntimeDelegate. Using the supplied RuntimeDelegate this
can be provided to JAX-RS in one of two ways:

\begin{enumerate}
    \item An instance of RuntimeDelegate can be instantiated and injected using its static method
    \lstinline{setInstance}. In this case the implementation is responsible for creating the instance; this option is
    intended for use with implementations based on IoC frameworks.
    \item The class to be used can be configured, see Section \ref{rdconfig}. In this case JAX-RS is responsible for
    instantiating an instance of the class and the configured class MUST have a public constructor which takes no
    arguments.
\end{enumerate}

Note that an implementation MAY supply an alternate implementation of the RuntimeDelegate API class (provided it passes
the TCK signature test and behaves according to the specification) that supports alternate means of locating a concrete
subclass.

A JAX-RS implementation may rely on a particular implementation of RuntimeDelegate being used -- applications SHOULD NOT
override the supplied RuntimeDelegate instance with an application-supplied alternative and doing so may cause
unexpected problems.

\section{Configuration}\label{rdconfig}

If not supplied by injection, the supplied RuntimeDelegate API class obtains the concrete implementation class using the
following algorithm. The steps listed below are performed in sequence and, at each step, at most one candidate
implementation class name will be produced. The implementation will then attempt to load the class with the given class
name using the current context class loader or, missing one, the \lstinline{java.lang.Class.forName(String)} method. As
soon as a step results in an implementation class being successfully loaded, the algorithm terminates.

\begin{enumerate}
    \item Use the Java SE class \lstinline{java.util.ServiceLoader} to attempt to load an implementation from
    \lstinline{META-INF\-/services/javax.ws.rs.ext.RuntimeDelegate}. Note that this may require more than one call to
    method \lstinline{ServiceLoader.load(Class, ClassLoader)} in order to try both the context class loader and the
    current class loader as explained above
    \footnote{Earlier versions of JAX-RS did not mandate the use \lstinline{ServiceLoader}. This backward-compatible
    change that started in JAX-RS 2.1 is to ensure forward compatibility with the Java SE 9 module system.}.
    \item If the \lstinline{$JAVA_HOME/lib/jaxrs.properties} file exists and it is readable by the
    \lstinline{java.util.Properties.load(InputStream)} method and it contains an entry whose key is
    \lstinline{javax.ws.rs.ext.RuntimeDelegate}, then the value of that entry is used as the name of the implementation
    class.
    \item If a system property with the name \lstinline{javax.ws.rs.ext.RuntimeDelegate} is defined, then its value is
    used as the name of the implementation class.
    \item Finally, a default implementation class name is used.
\end{enumerate}
